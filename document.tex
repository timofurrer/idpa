\documentclass[12pt,titlepage]{article}

\usepackage[T1]{fontenc}
\usepackage[utf8]{inputenc}
\usepackage[ngerman]{babel}
\usepackage{hyperref}
\usepackage{graphicx}
\usepackage{float}
\usepackage{longtable}
\usepackage{pdflscape}

% include pacakges
\usepackage[top=2cm, bottom=2cm, left=2.5cm, right=2.5cm, includefoot]{geometry} % geometrical layout
%\usepackage{fancyhdr} % header and footer
%\pagestyle{fancy}
\usepackage{enumerate} % enumerate lists

% header
%\lhead{Interdisziplinäre Projektarbeit}
%\rhead{Graf Dimitri und Furrer Timo}

% footer
%\cfoot{\thepage}

%\renewcommand{\headrulewidth}{0.4pt}
%\renewcommand{\footrulewidth}{0.4pt}

\begin{document}

\title{Anonymität im Netz - Ein Ding der Unmöglichkeit?}
\author{Dimitri Graf und Timo Furrer\\
  BMI4A\\
  Jutta Luecking\\
  Interdisziplinäre Projektarbeit\\
  GIBZ Zug}
\date{April 2014}
\maketitle

\tableofcontents
\newpage

\begin{abstract}
In dieser Arbeit wollen wir herausfinden, ob und mit welchen Mitteln es möglich ist, sich Anonymität im Netz zu schaffen. Welches Know-How braucht es dazu? Lohnt sich der Aufwand? 

Um diese Fragen zu beantworten, setzen wir uns mit drei Anwendungen auseinander, die online stattfinden: E-Mail-Verkehr, Surfen im Internet, Einsatz der Cloud. Parallel zu der theoretische Abhandlung wollen wir zudem Experimente mit Technologien durchführen, die diese Anwendung sicherer machen. Die Experimente werden in Form von Handbüchern dokumentiert und sollen Interessierten die Möglichkeit bieten, diese Techniken auszuprobieren.

Die Auseinandersetzung mit den Anwendungen und die Experimente haben gezeigt, dass es grundsätzlich möglich ist, sich Privatsphäre im Netz zu verschaffen. Es zeigte sich aber auch, dass es immer eine Frage des persönlichen Ehrgeizes und der Prioritäten ist. Letztendlich liegt es am Benutzer, sich für seine Privatsphäre einzusetzen.  

\end{abstract}

\input{$TEMP_DIR/sections.tex}

\listoffigures

\end{document}
