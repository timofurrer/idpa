\documentclass[12pt,titlepage]{article}

\usepackage[T1]{fontenc}
\usepackage[utf8]{inputenc}
\usepackage[ngerman]{babel}
\usepackage{hyperref}
\usepackage{graphicx}
\usepackage{float}
\usepackage{longtable}
\usepackage{pdflscape}

% include pacakges
\usepackage[top=2cm, bottom=2cm, left=2.5cm, right=2.5cm, includefoot]{geometry} % geometrical layout
%\usepackage{fancyhdr} % header and footer
%\pagestyle{fancy}
\usepackage{enumerate} % enumerate lists

% header
%\lhead{Interdisziplinäre Projektarbeit}
%\rhead{Graf Dimitri und Furrer Timo}

% footer
%\cfoot{\thepage}

%\renewcommand{\headrulewidth}{0.4pt}
%\renewcommand{\footrulewidth}{0.4pt}

\begin{document}

\title{Anonymität im Netz - Ein Ding der Unmöglichkeit?}
\author{Dimitri Graf und Timo Furrer\\
  BMI4A\\
  Jutta Lücking\\
  Interdisziplinäre Projektarbeit\\
  GIBZ Zug}
\date{April 2014}
\maketitle

\tableofcontents
\newpage

\begin{abstract}
In Laufe dieser Arbeit wollten wir herausfinden, ob es möglich ist, sich im Netz Anonymität zu schaffen und welche Mittel dazu benötigt werden. Welches Know-How braucht es dazu? Lohnt sich der Aufwand? Wo lauern überhaupt die Gefahren? 
\\
\\
Um diese Fragen zu beantworten, setzten wir uns mit drei Anwendungen auseinander, die heute Teil des täglichen Lebens sind: E-Mail-Verkehr, Surfen im Internet, Einsatz der Cloud. Parallel zu der theoretischen Abhandlung haben wir zudem Experimente mit Technologien durchgeführt, die diese Anwendung sicherer bzw. anonymer machen. Die Experimente wurden in Form von Handbüchern dokumentiert und sollen Interessierten die Möglichkeit bieten, diese Techniken selbst auszuprobieren.
\\
\\
Die Auseinandersetzung mit den Anwendungen und die Experimente haben gezeigt, dass es grundsätzlich möglich ist, sich Privatsphäre im Netz zu verschaffen. Es zeigte sich aber auch, dass es immer eine Frage des persönlichen Ehrgeizes und der Prioritäten ist. Letztendlich liegt es am Benutzer, sich für seine Privatsphäre einzusetzen.  

\end{abstract}

\input{$TEMP_DIR/sections.tex}

\listoffigures

\end{document}
