\documentclass[12pt,titlepage]{article}

\usepackage[T1]{fontenc}
\usepackage[utf8]{inputenc}
\usepackage[ngerman]{babel}
\usepackage{hyperref}
\usepackage{graphicx}
\usepackage{float}
\usepackage{longtable}
\usepackage{pdflscape}

% include pacakges
\usepackage[top=2cm, bottom=2cm, left=2.5cm, right=2.5cm, includefoot]{geometry} % geometrical layout
%\usepackage{fancyhdr} % header and footer
%\pagestyle{fancy}
\usepackage{enumerate} % enumerate lists

% header
%\lhead{Interdisziplinäre Projektarbeit}
%\rhead{Graf Dimitri und Furrer Timo}

% footer
%\cfoot{\thepage}

%\renewcommand{\headrulewidth}{0.4pt}
%\renewcommand{\footrulewidth}{0.4pt}

\begin{document}

\title{Anonymität im Netz - Ein Ding der Unmöglichkeit?}
\author{Dimitri Graf und Timo Furrer\\
  BMI4A\\
  Jutta Luecking\\
  Interdisziplinäre Projektarbeit\\
  GIBZ Zug}
\date{April 2014}
\maketitle

\tableofcontents
\newpage

\begin{abstract}
The abstract text goes here.
\end{abstract}

\input{$TEMP_DIR/sections.tex}

\end{document}
