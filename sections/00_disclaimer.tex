\section{Disclaimer}
Das Wort Disclaimer kommt aus dem Englischen von ``to disclaim'' was so viel bedeutet wie ``abstreiten'', ``widerrufen'', ``die Haftung ablehnen''
\footnote{\url{http://dict.leo.org/ende/index_de.html\#/search=to\%20disclaim}}.
Ein Disclaimer wird beispielsweise für E-Mails verwendet, um einem versehentlich falschen Empfänger mitzuteilen, den Inhalt der Mail wieder zu vergessen und sie an den Sender oder wahlweise den richtigen Empfänger weiterzuleiten. Auf Webseiten finden sich ebenfalls oft Disclaimer, wobei diese eine Distanzierung von möglicherweise verlinkten Seiten und Inhalten darstellen.
\footnote{\url{https://de.wikipedia.org/wiki/Disclaimer}}
Wir verwenden den Begriff in unserem Zusammenhang ein bisschen anders. In den folgenden Abschnitten möchten wir auf den Rahmen unserer Arbeit eingehen - Quellen, Aufbau und Veröffentlichung. Es also ein Disclaimer im Sinne eines Statements bzw. einer Erklärung und soll als Zusatzinformation betrachtet werden.

\subsection{Quellen}
Im Zuge der Erarbeitung dieser Arbeit haben wir vorwiegend auf Quellen aus dem World Wide Web (Internet) zurückgegriffen. Online-Enzyklopädien wie beispielsweise Wikipedia geniessen noch immer nicht den selben seriösen Ruf, wie gedruckte Fachliteratur und das teils auch zu Recht. Sie werden aber auch oft unterschätzt. Die Qualität von Wikipedia steigt kontinuierlich an und ist, was seine Grösse und Abdeckung betrifft, kaum zu übertreffen. Auch was unsere restlichen Quellen betrifft - Internetseiten, PDFs, News-Artikel - holten wir unser Wissen fast ausschliesslich online ab. Einerseits ist dies unsere Art, zu arbeiten und sind es auf diese Weise gewohnt. Andererseits sahen wir uns aber auch gezwungen, so und nicht anders vorzugehen. Denn in der IT-Welt ist das, was heute noch aktuell ist, morgen schon wieder von vorgestern. Fachbücher zu unseren Themen sind daher meist nicht mehr auf dem aktuellen stand oder gar nicht erst vorhanden. Wir führen diese Erklärung an dieser Stelle lediglich auf, um Sie als Leser darauf aufmerksam zu machen und ein gewisses Mass an Verständnis aufzubringen. Nicht zuletzt wollen wir auch erreichen, dass das Internet als Quelle des Wissens grössere Akzeptanz findet.

\subsection{Aufbau}
Dieses Dokument mit all ihren Kapitel stellt den Kern unserer Arbeit dar. Sie enthält Erklärungen, Definitionen, Übersicht und Details zu verschiedenen Konzepten und Technologien. All diese Dinge stellen jedoch reine Information dar, sind also nichts, was man direkt in die Praxis umsetzen könnte. Die Praxisbeispiele in Form von Anleitungen sind in drei separate Dokumente aufgeteilt, die in diesem ``Hauptdokument'' lediglich zusammengefasst und als Anhang aufgeführt sind. Diese Anleitungen oder Handbücher haben zum Ziel, einem Laien zu ermöglichen, die von uns beschriebenen Techniken und Vorgänge selbst anzuwenden und umzusetzen, denn oft ist der Weg dahin, auch für einen Laien, gar nicht so steinig, wie viele annehmen.

\subsection{Veröffentlichung}
Um diese Arbeit verwirklichen zu können haben wir viel an Wissen aus den Weiten des Internets holen können. All die Inhalte, von denen wir gebrauch machen konnten sind das Ergebnis der Arbeit eines dritten. Um der Gemeinschaft etwas zurückzugeben und das von uns erarbeitete Wissen zu teilen, wird diese Arbeit (Hauptdokument inkl. der drei Handbücher) im Internet veröffentlicht und ohne Entgelt zum Download angeboten. Auch stehen wir gerne für weiter Fragen sowohl zum Inhalt der Arbeit selbst als auch zum technischen Bereich gerne zur Verfügung.
Sie können die Arbeiten hier finden: .... xyz
