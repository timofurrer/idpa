\section{Disclaimer}
Das Wort Disclaimer kommt aus dem Englischen von ``to disclaim'', was so viel bedeutet wie ``abstreiten'', ``widerrufen'', ``die Haftung ablehnen''.\footnote{\url{http://dict.leo.org/ende/index_de.html\#/search=to\%20disclaim}}
Ein Disclaimer wird beispielsweise für E-Mails verwendet, um einen versehentlich falschen Empfänger zu bitten, den Inhalt der Mail wieder zu vergessen und sie an den Sender oder wahlweise den richtigen Empfänger weiterzuleiten. Auf Webseiten finden sich ebenfalls oft Disclaimer, wobei diese eine Distanzierung von möglicherweise verlinkten Seiten und Inhalten darstellen.\footnote{\url{https://de.wikipedia.org/wiki/Disclaimer}}
Wir verwenden den Begriff in unserem Zusammenhang ein bisschen anders. In den folgenden Abschnitten möchten wir auf den Rahmen unserer Arbeit eingehen - Quellen, Aufbau und Veröffentlichung. Es handelt sich hier also um einen Disclaimer im Sinne eines Statements bzw. einer Erklärung und soll als Zusatzinformation betrachtet werden.

% FIXME: ein bisschen anders ersetzen mit professionellerem Wort

\subsection{Quellen}
Im Zuge dieser Arbeit haben wir vorwiegend auf Quellen aus dem World Wide Web zurückgegriffen. Online-Enzyklopädien, wie beispielsweise Wikipedia, geniessen noch immer nicht den selben seriösen Ruf wie gedruckte Fachliteratur und das teils auch zu Recht. Sie werden aber auch oft unterschätzt. Die Qualität von  Wikipedia steigt kontinuierlich an und ist, was seine Grösse und Abdeckung angeht, kaum zu übertreffen. Auch was unsere restlichen Quellen betrifft - Internetseiten, PDFs, News-Artikel - haben wir unser Wissen fast ausschliesslich online bezogen. Einerseits ist dies unsere Art, zu arbeiten. Andererseits sahen wir uns aber auch gezwungen, so und nicht anders vorzugehen. Denn in der IT-Welt ist das, was heute noch aktuell ist, morgen schon wieder von vorgestern. Fachbücher zu unseren Themen sind daher meist nicht mehr auf dem aktuellen Stand oder gar nicht erst vorhanden. Wir führen diese Erklärung an dieser Stelle auf, um Sie als Leser auf diese Umstände aufmerksam zu machen und ein gewisses Mass an Verständnis zu erzeugen. Nicht zuletzt wollen wir auch erreichen, dass das Internet als Quelle des Wissens grössere Akzeptanz findet.

\subsection{Aufbau}
Dieses Dokument stellt den Kern unserer Arbeit dar. Es enthält Erklärungen, Definitionen, Übersicht und Details zu verschiedenen Konzepten und Technologien rund um das Thema ``Anonymität im Netz''. Der Inhalt dieses Dokumentes stellt also reine Information dar und ist nichts, was man direkt in die Praxis umsetzen könnte. Die Praxisbeispiele in Form von Anleitungen sind in drei separate Dokumente aufgeteilt, die in diesem ``Hauptdokument'' lediglich zusammengefasst und als Anhang aufgeführt sind. Die Handbücher sollen einem Laien ermöglichen, die von uns beschriebenen Techniken und Vorgänge selbst anzuwenden und umzusetzen. Denn oft ist der Weg dahin - selbst für einen Laien - gar nicht so steinig, wie oft angenommen wird. Jede der Anleitungen ist eine für sich abgeschlossene Einheit und kann unabhängig von den anderen Dokumenten verwendet werden.

\subsection{Veröffentlichung}
Um diese Arbeit zu verwirklichen, haben wir viel an Wissen aus den Weiten des Internets holen können. All die Inhalte, von denen wir Gebrauch gemacht haben, sind das Ergebnis der Arbeit eines Dritten. Um der Gemeinschaft etwas zurückzugeben und das von uns erarbeitete Wissen zu teilen, wird diese Arbeit (Hauptdokument inkl. der drei Handbücher) im Netz veröffentlicht und ohne Entgelt zum Download angeboten.

Sie können die Arbeiten hier finden: .... xyz
