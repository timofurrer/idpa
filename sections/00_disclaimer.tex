\section{Formelle Hinweise}

\subsection{Quellen}
Im Zuge dieser Arbeit haben wir vorwiegend auf Quellen aus dem World Wide Web zurückgegriffen. Gerade Online-Enzyklopädien geniessen noch immer nicht den selben seriösen Ruf wie gedruckte Fachliteratur und das teils auch zu Recht. Sie werden aber auch oft unterschätzt. Wikipedia zum Beispiel hat inzwischen einen hohen Qualitätsgrad erreicht und ist, was seine Grösse und Abdeckung angeht, kaum zu übertreffen. Auch was unsere restlichen Quellen betrifft - Internetseiten, PDFs, News-Artikel - haben wir unser Wissen fast ausschliesslich online bezogen. Einerseits ist dies unsere Art zu arbeiten. Andererseits sahen wir uns aber auch gezwungen, so und nicht anders vorzugehen. Denn in der IT-Welt ist das, was heute noch aktuell ist, morgen schon wieder von vorgestern. Fachbücher zu den von uns behandelten Themen sind daher meist nicht mehr auf dem aktuellen Stand oder gar nicht erst vorhanden. Wir führen diese Erklärung an dieser Stelle auf, um den Leser auf diese Umstände aufmerksam zu machen und ein gewisses Mass an Verständnis zu erzeugen. Nicht zuletzt wollen wir auch erreichen, dass das Internet als Quelle des Wissens grössere Akzeptanz findet.

\subsection{Aufbau}
Dieses Dokument stellt den Kern unserer Arbeit dar. Es beinhaltet die theoretische Abhandlung zur Fragestellung \textit{Anonymität im Netz - Ein Ding der Unmöglichkeit?} und ist in verschiedene Themengebiete unterteilt. Als Teil der Projektarbeit haben wir ausserdem mehrere Experimente durchgeführt, die direkten Bezug zu in diesem Dokument behandelten Themen haben. Diese Experimente sind stark praxisbezogen und  wurden von uns in Form von Handbüchern dokumentiert. Die Handbücher sollen dem Anwender ermöglichen, die von uns beschriebenen Techniken zur Wahrung der Anonymität selbst anzuwenden, denn oft ist der Weg dahin, selbst für einen Laien, gar nicht so steinig wie vielleicht angenommen. Jedes Handbuch ist eine für sich abgeschlossene Einheit und kann unabhängig von den anderen Dokumenten verwendet werden. Die Handbücher befinden sich im Anhang zu diesem Dokument und können alternativ einzeln heruntergeladen werden. Mehr dazu im nächsten Abschnitt.

\subsection{Veröffentlichung}
Um diese Arbeit zu verwirklichen, haben wir viel an Wissen aus den Weiten des Internets holen können. All die Inhalte, von denen wir Gebrauch gemacht haben, sind das Ergebnis der Arbeit eines Dritten. Um das von uns erarbeitete Wissen zu teilen und der Gemeinschaft etwas zurückzugeben, werden das Hauptdokument und die Handbücher im Netz veröffentlicht und ohne Entgelt zum Download angeboten.
Die Arbeiten können unter folgender Adresse heruntergeladen werden: \url{http://timofurrer.github.io/idpa}
