\section{Rückblick}

\subsection{Sicherer Mailverkehr}

Wie uns der Versuch einen eigenen Mailserver einzurichten gezeigt hat, 
ist es für einen Laien sehr schwierig oder gar unmöglich einen eigenen Mailserver zu betreiben.
Das dafür nötige Know-How kann zwar mit viel Aufwand erarbeitet werden, jedoch lohnt sich dies in der Praxis oft nicht. 
Die bessere Alternative wäre einen sicheren und vertrauenswürdigen Mailprovider auszumachen. 
Hierbei ist ein Provider hier in der Schweiz sicher keine schlechte Wahl. 
Ein grosses Kriterium bleibt aber bestehen - Die volle Kontrolle über den Mailserver hat nur der Provider selbst!

Eine andere Möglichkeit wäre es eine Person oder eine Gruppe von Leuten mit gleichem Interesse zu finden, die es ermöglich,
mit finanziellen Mitteln und technischem Know-How, einen sicheren Mailserver zu betreiben.

\subsection{Anonym Surfen}

Anonymes Surfen ist in der heutigen Zeit sicher von grosser Bedeutung. 
Nirgends ist die Gefahr der Privatsphärenverletzung so gross und akut. 
Die Karten liegen vollkommen offen - Jeder Webseitenaufruf kann auf einen bestimmten Benutzer zurückgeführt werden.
Das in dem Experiment verwendete Tor-Netzwerk ist eine gute Möglichkeit die Privatsphäre zu schützen.
Der Einsatz von Tor ist aber nur effektiv, wenn es korrekt eingerichtet, angewendet und mit weiteren techniken und Programmen ausgestattet wird.
Wenn man sich bei der Einrichtung von Tor nicht hundert Prozent sicher ist, sollte man eine Alternative vorziehen.
Eine kurze Internetrecherche wird eine Menge Alternativen zeigen.

\subsection{Private Daten in der Cloud}

Heutzutage werden Daten immer häufiger online gespeichert, also auf irgendwelchen Servern von grossen Providern. 
Private Daten könnten also potenziell eingesehen werden. Das Experiment, eine eigene Cloud einzurichten, war das simpelste von allen. 
Es ist für einen Laien mit ein wenig Recherche gut umzusetzen und bringt einen grossen Ertrag für verhältnismässig wenig Aufwand.
Das verwendete Programm ownCloud hat einen grossen Funktionalitätsumfang und kommt somit sehr nahe an grosse, kommerzielle Anbieter.
Wie bei dem Mailverkehr gäbe es auch hier die Möglichkeit, einen kleineren, vertrauenswürdigen Provider zu wählen.
Die Souveränität über die Verwaltung der Daten wäre dann aber nicht gegeben.

\subsection{Fazit}

Die Experimente haben gezeigt, dass es definitiv möglich ist, sich im Internet mehr Privatsphäre zu schaffen.
Es ist letzendlich eine Frage des Aufwands und der persönlichen Prioritäten. Jemand mit technischen Vorkenntnissen und genügend Ehrgeiz hat die Möglichkeit mit eigenen Mitteln seine Privatsphäre zu schützen. Er kann zu grossen Teilen Einfluss nehmen, auf die persönlichen Daten, die im Internet kursieren. Hat man aber nicht das technische Know-How oder den Ehrgeiz, sich dieses anzueignen, muss man sich auf Dienstleister verlassen. Aber auch da braucht es ein gewisses Mass an Eigeninitiative, um einen vertrauenswürdigen Provider ausfindig zu machen. Sofern es denn einen gibt.
