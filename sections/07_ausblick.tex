\section{Rückblick}

\subsection{Sicherer Mailverkehr}
Den Mailverkehr sicher zu gestalten sollte eigentlich eine Selbstverständlichkeit sein. Schliesslich geht es um private Nachrichten, die mehr oder weniger heikle Informationen beinhalten.
Wie der Versuch gezeigt hat, ist es für einen Laien sehr schwierig oder gar unmöglich, einen eigenen Mail-Server einzurichten und zu betreiben.
Das dafür nötige Know-How kann zwar mit viel Aufwand erarbeitet werden, jedoch lohnt sich dies in der Praxis oft nicht. 
\\
Die bessere bzw. einfachere Alternative ist, einen sicheren und vertrauenswürdigen Mailprovider ausfindig zu machen. 
Ein Provider in der Schweiz ist hierbei sicher keine schlechte Wahl. 
Ein grosser Kritikpunkt bleibt aber bestehen: Die volle Kontrolle über den Mail-Server hat nur der Provider selbst!
\\
Eine dritte Möglichkeit wäre, eine Person oder eine Gruppe von Leuten mit gleichem Interesse zu finden. In der Gruppe ist es einfacher die finanziellen Mittel und das technische Know-How anzuschaffen, das es braucht, um einen sicheren Mail-Server zu betreiben.

\subsection{Anonym Surfen}
Das Risiko der Privatsphärenverletzung ist im World Wide Web beinahe omnipräsent. 
Die Karten liegen vollkommen offen - jeder Webseitenaufruf kann auf einen bestimmten Benutzer zurückgeführt werden.
Das in dem Experiment verwendete Tor-Netzwerk ist eine gute Möglichkeit, die Privatsphäre im Netz zu erhöhen.
Tor ist aber nur effektiv, wenn es korrekt eingerichtet, angewendet und mit weiteren Techniken und Programmen kombiniert wird.
Wenn man sich bei der Einrichtung von Tor nicht hundert Prozent sicher ist, sollte man deshalb lieber auf Alternativen zurückgreifen..
\\
Es gibt beispielsweise Erweiterungen für den Browser, die das Surf-Erlebnis sicherer gestalten. Eine kurze Recherche im Internet wird viele Möglichkeiten aufzeigen, wie genau man sich im Web anonymer fortbewegen kann.

\subsection{Private Daten in der Cloud}
Heutzutage werden Daten immer häufiger online gespeichert, also auf irgendwelchen Servern von grossen Providern abgelegt. 
Private, personenbezogene Daten könnten also potenziell eingesehen werden. Das Experiment, eine eigene Cloud einzurichten, hat gezeigt, dass es möglich ist, seine Daten selbst zu verwalten, ohne von grossen Dienstleistern abhängig zu sein.
Es ist für einen Laien mit ein wenig Recherche gut umsetzbar und bringt einen grossen Ertrag für verhältnismässig wenig Aufwand.
Das verwendete Programm ownCloud hat einen grossen Funktionalitätsumfang und kommt somit sehr nahe an grosse, kommerzielle Anbieter.
\\
Ist einem der Aufwand trotzdem zu gross, gäbe es auch hier die Möglichkeit, einen vertrauenswürdigen Anbieter zu wählen, der einem die gwünschte Funktionalität zur Verfügung stellt.
Die Souveränität über die Daten wäre dann aber nicht mehr gegeben.

\subsection{Fazit}
Die Experimente haben gezeigt, dass es definitiv möglich ist, sich im Internet mehr Privatsphäre zu schaffen.
Es ist letzendlich eine Frage des Aufwands und der persönlichen Prioritäten. Jemand mit technischen Vorkenntnissen und genügend Ehrgeiz hat die Möglichkeit mit eigenen Mitteln seine Privatsphäre zu schützen. Hat man aber nicht das technische Know-How oder den Ehrgeiz, sich dieses anzueignen, muss man sich auf Dienstleister verlassen. Aber auch da braucht es ein gewisses Mass an Eigeninitiative, um einen vertrauenswürdigen Provider ausfindig zu machen. Sofern es denn einen gibt.
Die Frage \textit{Anonymität - Ein Ding der Unmöglichkeit?} kann schlussendlich nicht mit \textit{Ja} oder \textit{Nein} beantwortet werden. Vielmehr geht es darum, sich bewusst im Netz fortzubewegen und die Möglichkeiten wahrzunehmen, die sich zum Schutz der Privatsphäre bieten.
