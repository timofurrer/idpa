\newpage
\section{Einleitung}

\subsection{Was bisher geschah?}
Anfang Juni 2013 wurden erste Dokumente an die Öffentlichkeit gebracht, die die massenhafte Überwachung von Telefongesprächen von US-Amerikanern belegen. Gemäss den Dokumenten steckte dahinter der Amerikanischen Auslandsgeheimdienstes NSA (National Security Agency). In den darauf folgenden Tagen, Wochen und Monaten wurden immer mehr Dokumente und Zeitungsartikel veröffentlicht, die die wahren Ausmasse der globalen Überwachung grosser Geheimdienste ans Licht brachten. Die Publikationen dauern noch immer an. Kurz nach den ersten veröffentlichungen bekennte sich ein junger Mann namens Edward Snowden als den entscheidenden Informanten, der die Unmengen an Dokumenten in seinen Besitz gebracht und an Journalisten seines Vertrauens weitergegeben hat. Es wurde bis heute nicht klar, welche Menge an Informationen Edward Snowden an sich bringen und weitergeben konnte und wer alles Kopien davon hat. Die betroffenen Geheimdienste, darunter die NSA und der britische GCHQ, und die darüber stehenden Regierungen bemühen sich seit den ersten Veröffentlichungen um Schadensbegrenzung. Dies ist aber nur Ansatzweise gelungen ist. Nun ist schon seit längerer Zeit eine rege Diskussion auf globaler Ebene im Gange, die sich mit den Tätigkeiten von Geheimdiensten, der Privatsphäre jedes einzelnen Bürgers und der Sicherheit eines jeden Landes beschäftigt. Barack Obama soll bei einer Rede, bei der er die Tätigkeiten der NSA verteidigte, gesagt haben: ``Man kann nicht 100 Prozent Sicherheit und 100 Prozent Privatsphäre und null Unannehmlichkeiten haben.''
\footnote{\url{https://de.wikipedia.org/wiki/\%C3\%9Cberwachungs-_und_Spionageaff\%C3\%A4re_2013\#Politik}}
Wir stehen dieser Aussage kritisch gegenüber und wollten uns näher mit der Thematik ``Anonymität im Netz'' auseinandersetzen.

\subsection{Motivation}
Als angehende Informatiker haben wir die Enthüllungen und damit einhergehenden Diskussionen mit grossem Interesse und wachsender Sorge verfolgt. Durch unser technisches Verständnis konnten wir uns möglicherweise auch mehr darunter vorstellen als jemand, der Technik hauptsächlich anwendet, ohne sie gross zu hinterfragen oder deren innere Funktionsweise zu kennen. Wir sind der Meinung, dass ein jeder Bürger das Recht auf Privatsphäre hat. Um diese zu Schützen, braucht es heutzutage gewisse Massnahmen, die man so bis anhin vielleicht nicht anwandte, geschweige denn kannte. Die Frage, die wir uns stellen ist, ob Anonymität im Netz heutzutage überhaupt möglich ist und falls die Antwort auf diese Frage ``Ja'' lautet, zu welchem Preis? Welches technische ``Know-How" muss man besitzen, um verstehen zu können, an welchen Punkten eine ungewollte Überwachung stattfinden kann. Und wie gross ist der Aufwand, sich an eben diesen Stellen zu schützen. 
% FIXME: Eine grosse Frage die sich auch stellt, ist, ob der Mensch und die Gier der Kontrolle das Problem ist oder die Technik die einen dazu verleiten lässt zu kontrollieren? 
% --> philosophische Frage, die in unserer Arbeit nicht abgedeckt wird


\subsection{Ziel}
Unser Ziel ist es, die Antworten auf die oben genannten Fragen zu finden und im Zuge unserer Recherchen Anleitungen zu erstellen, die es einem Laien ermöglichen sollen, sich selbst im Netz die bestmögliche Privatsphäre zu schaffen. Wir wählten drei Anwendungen aus dem täglichen Leben, die online, also über das Internet, erfolgen: Browsen im Internet, E-Mail-Verkehr und Arbeiten mit der Cloud. Jede dieser Anwendungen wird genauer unter die Lupe genommen. Der Schwerpunkt liegt dabei auf dem Einsatz von Technologien, die die drei Anwendungen sicherer machen und die Privatsphäre im Netz schützen. Diese Technologien werden wir ausführlich beschreiben und auf Vor- und Nachteile eingehen. Zusätzlich wollen wir für jede der drei Technologien ein Handbuch erstellen, das dem Leser ermöglichen soll, selbst Gebrauch von ihnen zu machen. Letzten Endes soll diese Arbeit den Leser sensibilisieren für die Thematik ``Anonymität im Netz'' und im Mittel in die Hand geben, sich diese zu sichern.
