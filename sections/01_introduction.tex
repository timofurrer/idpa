\newpage
\section{Einleitung}

\subsection{Was bisher geschah}
Anfang Juni 2013 gelangten erste Dokumente an die Öffentlichkeit, die belegen, dass der amerikanische Auslandsgeheimdienst NSA (National Security Agency) im grossen Stil die Telefongespräche von US-Amerikanern überwacht hatte. In den darauf folgenden Tagen, Wochen und Monaten wurden immer mehr Dokumente und Zeitungsartikel veröffentlicht, die die wahren Ausmasse der Überwachung im Netz seitens der Geheimdienste ans Licht brachten. Und der Strom an Publikationen reisst noch immer nicht ab. Kurz nach den ersten Veröffentlichungen outete sich ein junger Mann namens Edward Snowden als den entscheidenden Informanten, der die Unmengen an Dokumenten in seinen Besitz gebracht und an Journalisten seines Vertrauens weitergegeben hatte. Es ist bis heute nicht offiziell klar, welche Menge an Informationen der ehemalige NSA-Mitarbeiter an sich bringen und an wen er Kopien weitergeben konnte. Die betroffenen Geheimdienste, darunter die NSA und der britische GCHQ (Government Communications Headquarters), und die darüber stehenden Regierungen bemühen sich seit den ersten Veröffentlichungenn um Schadensbegrenzung. Nun ist schon seit längerer Zeit eine rege Diskussion auf globaler Ebene im Gange, die sich mit den Tätigkeiten von Geheimdiensten, der Privatsphäre der Bürger und der Sicherheit eines jeden Landes beschäftigt. Barack Obama soll bei einer Rede, bei der er die Tätigkeiten der NSA verteidigte, gesagt haben: ``Man kann nicht 100 Prozent Sicherheit und 100 Prozent Privatsphäre und null Unannehmlichkeiten haben.''\footnote{\url{https://de.wikipedia.org/wiki/\%C3\%9Cberwachungs-_und_Spionageaff\%C3\%A4re_2013\#Politik}}
Wir stehen dieser Aussage kritisch gegenüber und wollten uns näher mit der Thematik ``Anonymität im Netz'' auseinandersetzen.

% FIXME: Zusammenhang zum WWW herstellen

\subsection{Motivation}
Als angehende Informatiker haben wir die Enthüllungen und die damit einhergehenden Diskussionen mit grossem Interesse und wachsender Sorge verfolgt. Durch unser technisches Verständnis konnten wir uns möglicherweise auch mehr darunter vorstellen als jemand, der Technik hauptsächlich anwendet, ohne sie zu hinterfragen oder deren innere Funktionsweise zu kennen. Wir sind der Meinung, dass ein jeder Bürger das Recht auf Privatsphäre hat. Um diese zu schützen, braucht es heutzutage gewisse Massnahmen, die man so bis anhin vielleicht nicht angewandt, geschweige denn gekannt hat. Die Frage, die wir uns stellen, ist, ob Anonymität im Netz heutzutage überhaupt noch möglich ist. Welches technische ``Know-How'' muss man besitzen, um zu verstehen, an welchen Punkten eine ungewollte Überwachung stattfinden kann? Und wie gross ist der Aufwand, sich an eben diesen Stellen zu schützen?

\subsection{Ziel}
Unser Ziel ist es, die Antworten auf die oben genannten Fragen zu finden. Wir wählten drei Anwendungen aus dem täglichen Leben, die online, also über das Internet, erfolgen: Browsen im Internet, E-Mail-Verkehr und Arbeiten mit der Cloud. Jede dieser Anwendungen wird genauer unter die Lupe genommen. Der Schwerpunkt liegt dabei auf dem Einsatz von Technologien, die die drei Anwendungen sicherer machen und die Privatsphäre im Netz schützen. Diese Technologien werden wir ausführlich beschreiben und auf ihre Vor- und Nachteile eingehen. Zusätzlich wollen wir für jede der drei Technologien ein Handbuch erstellen, das dem Leser ermöglicht, selbst Gebrauch von ihnen zu machen. Letzten Endes soll diese Arbeit den Leser für die Thematik ``Anonymität im Netz'' sensibilisieren und ihm Mittel in die Hand geben, sich diese zu sichern.
