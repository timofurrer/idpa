\newpage
\section{Einleitung}

\subsection{Was bisher geschah}
Anfang Juni 2013 gelangten erste Dokumente an die Öffentlichkeit, die belegen, dass der amerikanische Auslandsgeheimdienst NSA (National Security Agency) im grossen Stil die Telefongespräche von US-Amerikanern überwacht hatte.\footnote{Link 1, Linkverzeichnis, heise}
In den darauf folgenden Tagen, Wochen und Monaten wurden immer mehr Dokumente und Zeitungsartikel veröffentlicht, die die wahren Ausmasse der Überwachung digitaler Medien seitens der Geheimdienste ans Licht brachten. Und der Strom an Publikationen reisst noch immer nicht ab. Kurz nach den ersten Veröffentlichungen outete sich ein junger Mann namens Edward Snowden als entscheidender Informant, der die vielen Dokumente in seinen Besitz gebracht und an Journalisten seines Vertrauens weitergegeben hatte. Es ist bis heute nicht offiziell klar, welche Menge an Informationen der ehemalige NSA-Mitarbeiter an sich bringen und an wen er Kopien weitergeben konnte.\footnote{Link 2, Linkverzeichnis, Spiegel}
Die betroffenen Geheimdienste, darunter die NSA und der britische GCHQ (Government Communications Headquarters), und die darüber stehenden Regierungen bemühen sich seit den ersten Veröffentlichungen um Schadensbegrenzung. Nun ist schon seit geraumer Zeit eine rege Diskussion auf globaler Ebene im Gange, die sich mit den Tätigkeiten von Geheimdiensten, der Privatsphäre der Bürger und der Sicherheit eines jeden Landes beschäftigt. Barack Obama soll bei einer Rede, bei der er die Tätigkeiten der NSA verteidigte, gesagt haben: \textit{Man kann nicht 100 Prozent Sicherheit und 100 Prozent Privatsphäre und null Unannehmlichkeiten haben.}\footnote{Link 3, Linkverzeichnis, Wikipedia}
Wir stehen dieser Aussage kritisch gegenüber und wollten uns näher mit der Thematik \textit{Anonymität im Netz} auseinandersetzen.

\subsection{Motivation}
Als angehende Informatiker haben wir die Enthüllungen und die damit einhergehenden Diskussionen mit grossem Interesse und wachsendem Unmut verfolgt. Durch unser technisches Verständnis konnten wir uns möglicherweise mehr darunter vorstellen als jemand, der Technik hauptsächlich anwendet, ohne sie zu hinterfragen oder deren innere Funktionsweise zu kennen. Wir sind der Meinung, dass ein jeder Bürger das Recht auf Privatsphäre hat. Um diese zu schützen, braucht es heutzutage gewisse Massnahmen, die man so bis anhin vielleicht nicht angewandt, geschweige denn gekannt hat. Ist Anonymität im Netz heutzutage überhaupt noch möglich? Welches technische \textit{Know-How} muss man besitzen, um zu verstehen, an welchen Punkten eine ungewollte Überwachung stattfinden kann? Und wie gross ist der Aufwand, sich an eben diesen Stellen zu schützen?

\subsection{Ziel}
Unser Ziel ist es, im Rahmen dieser Arbeit Antworten auf die oben genannten Fragen zu finden. Wir wählten dafür drei Anwendungen aus dem täglichen Leben, die online, also über das Internet, erfolgen: E-Mail-Verkehr, Arbeiten mit der Cloud und Browsen im Internet. Jede dieser Anwendungen wird genauer unter die Lupe genommen und in einen grösseren Zusammenhang gestellt. Der Schwerpunkt liegt dabei auf dem Einsatz von Techniken bzw. Technologien, die diese Anwendungen sicherer machen und die Privatsphäre im Netz schützen oder zumindest erhöhen. Zusätzlich wollen wir die Technologien in praxisnahen Experimenten selbst testen und parallel dazu Handbücher erstellen, die dem Anwender ermöglichen, selbst Gebrauch von ihnen zu machen. Letzten Endes soll diese Arbeit den Leser für die Thematik \textit{Anonymität im Netz} sensibilisieren und ihm Mittel in die Hand geben, sich diese zu sichern.
