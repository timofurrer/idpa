\section{E-Mails und deren Tücken}
Täglich werden heute bis zu 180 Milliarden E-Mails versendet. Eine gewaltige Menge, wenn man bedenkt, dass es ein wenig mehr als sieben Milliarden Menschen auf diesem Planeten gibt.
Doch nur wenige E-Mail-Nutzer machen sich wohl Gedanken darüber, was denn eigentlich hinter einer E-Mail steckt. Wohin geht sie genau? Wer kann diese E-Mail theoretisch lesen? Wo wird diese Nachricht überall gespeichert? Und für wie lange? Gibt es Kopien? Backups? Auf anderen Servern?
All diese Fragen sind sehr wichtig, wenn man vertrauliche Daten per Mail versendet. Doch auch wenn die Nachricht nicht von grosser Bedeutung ist, möchte man doch eine gewisse Privatsphäre wahren, oder?
Was kann man also als „0815“ Nutzer tun, um seine eigenen E-Mails möglichst sicher zu versenden, zu speichern und zu empfangen?
\\
Im nächsten Kapitel wird versucht auf diese Fragen eine Antwort zu finden. Zuerst einmal muss man aber verstehen, wie das genau funktioniert, mit den „Mails“.

\subsection{Funktionsweise des Mailverkehrs}

\begin{figure}[H]
\centering
\noindent\includegraphics[scale=0.6]{images/email_function}
\caption{Funktionsweise Mailverkehr}
\end{figure}

Eine E-Mail passiert von der Erzeugung beim Sender bis zur Abgabe beim Empfänger mehrere Posten. Um die Funktionsweise des E-Mail-Verkehrs beschreiben zu können, müssen diese Posten zuerst erklärt werden:

\begin{itemize}
\item Mail-Client: Beim Client handelt es sich um das Mail-Programm, welches E-Mails senden und empfangen kann. Sowohl der Sender als auch der Empfänger sind auf ein solches angewiesen.
\item Mail Transfer Agent (MTA): Der Mail Transfer Agent ist für den Transport der E-Mail verantwortlich.
\footnote{Link 32, Linkverzeichnis, Wikipedia}
\item Mail Delivery Agent (MDA): Der Mail Delivery Agent hat die Aufgabe, die E-Mail beim Empfänger abzuliefern, sobald diese vom Mail Transport Agent übermittelt wurde.
\footnote{Link 33, Linkverzeichnis, Wikipedia}
\end{itemize}

Bei MTA und MDA handelt es sich um Computer-Programme, welche typischerweise auf den Servern (Computern) des jeweiligen Providers launfen.
\\
\\
Ablauf des E-Mail-Verkehrs:

\begin{enumerate}
\item Der Sender verfasst mit Hilfe des Clients eine Mail für einen bestimmten Empfänger drückt anschliessend auf \textit{Senden}.
\item Der Client übermittelt das E-Mail bei aktiver Internetverbindung an den ersten MTA.
\item Das Mail wird nun von MTA zu MTA geschickt, bis derjenige des Empfängers gefunden wird.
\item Der MTA des Empfängers übermittelt das Mail an den MDA, welcher das Mail aufbewahrt.
\item Der Client des Empfängers holt das Mail beim MDA ab und stellt es dem Empfänger zur Verfügung.
\end{enumerate}

Je nach verwendetem Protokoll wird dem Empfänger eine Kopie des Mails zugestellt (IMAP), wobei das Original auf dem Server bleibt, oder er erhält das Original (POP3).
\\
\\
IMAP (Internet Message Access Protocol) hat den Vorteil, das man im Falle eines lokalen Datenverlusts noch eine Kopie des Mails auf dem Server hat. Es eignet sich zudem sehr gut für die parallele Nutzung von mehreren Geräten (Computer, Tablet, Smartphone), da die E-Mails jeweils synchronisiert (abgeglichen) werden.
\footnote{Link 35, Linkverzeichnis, Wikipedia}
\\
\\
POP3 (Post Office Protocol 3) bietet diese Möglichkeiten nicht. Es eignet sich aber für Leute, die gerne selbst bestimmen, wo ihre E-Mails aufbewahrt werden. Seit den Snowden-Enthüllungen hat dieses Argument wieder an Bedeutung gewonnen.

\subsection{Mein Provider und seine Server}
Die erste Entscheidung, die man treffen muss, um überhaupt eine E-Mail senden zu können, noch bevor man überhautp seinen Mail-Client startet, ist es, sich einen passenden Provider auszusuchen. Ein Provider ist meist eine Firma, die seinen Kunden eine Mail Adresse zur Verfügung stellt. Dies kann z.B. Google mit Gmail, Microsoft mit Hotmail oder eine Firma wie GMX sein. Was ein solcher Provider also tut, ist es irgendwo eine E-Mail Adresse für den Kunden zu konfigurieren. Dazu gehört einmal eine Adresse zum Mail Transport Agent und eine Adresse zum Mail Delivery Agent Server.
Diese beiden Server sind nicht nur auf den Bezug für die Mailfunktionalität im allgemeinen von Bedeutung. Nein, denn dort, auf diesen Servern, unter der Kontrolle des jeweiligen Providers, geht jede einzelne E-Mail, die versendet oder empfangen wird durch, was bedeutet, dass dort theoretisch ein Mitschnitt aller Nachrichten gemacht werden kann. Auch macht der Provider, natürlich nur zum besten seiner Kunden, Backups deren Postfächer, auf dem MDA. Das mag ja schön und gut sein, aber wer, ausser meinerselbst kann auch noch auf dieses Postfach oder die Backup-Daten zugreifen? Die Systemadministratoren vom Provider? PR-Firmen? Datenanalysten? Geheimdienste? Wer weiss …
Wie man sehen kann, hat man, abgesehen vom Vertrauen, im Prinzip keine Ahnung, wer alles auf seine E-Mails zugreifen kann.

\subsection{E-Mails verschlüsseln}
Doch es gibt ein paar Dinge, die man als E-Mail Nutzer tun kann, um seine Post einigermassen vor ungewollten Blicken schützen zu können.
So gibt es mehrere Verfahren um E-Mails zu verschlüsseln. Eines der beliebtesten und wohl meist verbreitetsten ist PGP (Pretty Good Privacy) oder deren Weiterentwicklung GPG (GNU Privacy Guard).
Bei beiden handelt es sich um ein Public-Key-Verfahren mit asymmetrischer Verschlüsselung. Das heisst, dass es einmal einen öffentlichen Schlüssel gibt, mit dem jeder beliebige Daten für den Empfänger verschlüsseln und signieren kann und einen privaten geheimen Schlüssel der vom Empfänger benutzt wird um die Nachricht, die mit seinem öffentlichen Schlüssel verschlüsselt wurde, zu entschlüsseln.
Solange man seinen privaten Schlüssel geheim behält, ist es wahnsinnig schwierig, sowie sehr Rechenleistungs- und Zeitinsensiv einen solch verschlüsselte Nachricht zu knacken.
Es gibt eine Menge an Erweiterungen für die gängisten Mail-Clients um diese Verschlüsselungsfunktion nach zu rüsten.

\subsection{Ich als Provider}
Wie wir gesehen haben, macht ein Provider eigentlich nicht viel aus. Was dieser lediglich zu tun hat, damit man E-Mails versenden und empfangen kann, ist es einen MTA und einen MDA zur Verfügung zu stellen, sowie die E-Mail Adresse korrekt zu konfigureren, damit die Mails auch ihren richtigen Weg gehen. Natürlich müssen diese beiden Server am Internet angehängt sein und dauerhaft laufen.
Jeder E-Mail Provider hat zudem seine eigene Domain, wie zum Beispiel Google's \textit{gmail.com} oder Swisscom's \textit{bluewin.ch}. Die Kunden E-Mail Adressen enden also immer auf jene Domains: \textit{max@gmail.com}, \textit{muster@bluewin.ch}. Natürlich braucht man auch selbst eine solche Domain. Diese muss nicht zwingend eine Top-Level-Domain \footnote{Link 36, Linkverzeichnis, Wikipedia} sein. Auch eine kostenlose Domain mit der Endung \textit{.ch.vu} würde genügen. Dieser Domain braucht man dann nur noch einen DNS-Server \footnote{\url{http://de.wikipedia.org/wiki/Domain\_Name\_System}} zuweisen, der unsere MDA Server IP kennt. Ein Problem gibt es da aber noch und zwar, dass normale Haushaulte vom Internetprovider keine statische IP Adresse bekommen, was bedeutet, dass diese ständig ändern könnte und so der DNS-Server nicht mehr weiss wohin er jetzt den Datenverkehr weiterleiten muss. Hier gibt es zwei Lösungen, um dieses Problem zu vermeiden: Man kann beim Provider eine meist kostenpflichtige statische IP Adresse verlangen oder man registriert sich bei einem dynamischen DNS Anbieter. Dieser bietet eine Schnittstelle zur Verfügung, über die man seine Server IP Adresse, immer wenn diese ändert, dem DNS mitteilen kann. Somit ist gegeben, dass der DNS Server stets die aktuelle IP Adresse des Servers kennt ohne dass man auf eine kostenpflichtige statische IP Adresse zurückgreifen muss.

\subsection{Vor- und Nachteile}
Natürlich gibt es nicht nur Vorteile wenn man sein eigener E-Mail Provider ist. Folgende Auflistung soll hier eine Übersicht schaffen. \\ \\

\textbf{Vorteile:}
\begin{itemize}
    \item Volle Kontrolle
    \item Verschlüsselung
    \item Eigener Spam und Virenfilter
\end{itemize}

\textbf{Nachteile:}
\begin{itemize}
    \item Hardware- und Unterhaltungs-Kosten
    \item Benötigte Zeit
\end{itemize}

\subsection{Vertrauenswürdige Provider?}
Die Frage die sich natürlich auch stellt ist, ob es denn bereits vertrauenswürdige und sichere E-Mail Provider gibt. \\
Leider ist es sehr schwierig oder gar unmöglich eine Antwort auf diese Frage zu finden, denn Tatsache ist, dass in der Theorie jeder Provider gehackt werden kann, was natürlich bedeutet, dass eben ein solcher Hacker an die Daten gelangen kann. \\
Also bleibt einen eigentlich nur noch das Vertrauen. Vertrauen in den Provider, dass dieser bestrebt ist, jede mögliche Massnahme zu treffen, um die höchsten Sicherheitsstandards einzuhalten. \\
Ein Beispiel hierfür ist zum Beispiel \textit{NEOMAILBOX}\footnote{Link 37, Linkverzeichnis, Neomailbox}. Diese Firma stellt einen kostenpflichtigen Service zur Verfügung, der einen schnellen, sicheren und anonymen E-Mail Dienst verspricht. Zudem soll der Service mit einem Top Spam- und Virenfilter ausgestattet sein.\footnote{Link 38, Linkverzeichnis, Neomailbox}
