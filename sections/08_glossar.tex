\section{Glossar - Fachbegriffe}

\begin{tabular*}{\textwidth}{|p{0.2\textwidth}|p{0.75\textwidth}|}
  \hline
  \cellcolor{gray}Begriff & \cellcolor{gray}Bedeutung \\ \hline
  WWW & Word Wide Web \\
      & Bezeichnung für das Internet \\ \hline
  PDF & Portable Document Format \\
      & Plattform unabhängiges Dateiformat für Dokumente \\ \hline
  NSA & National Security Agency \\
      & Der Auslandsgeheimdienst der Vereinigten Staaten \\ \hline
  GCHQ & Government Communications Headquarters \\
       & Britische Regierungsbehörde, zuständig für Kryptographie, Datenübertragung und Fernmeldeaufklärung \\ \hline
  Cloud & Internet Datenspeicher \\ \hline
  MAC Adresse & Eindeutige Identifikationsnummer einer Netzwerkkarte \\ \hline
  SSL & Secure Sockets Layer \\
      & Verschlüsselungsverfahren \\ \hline
  IT & Information Technology \\ \hline
  Stasi & Ministerium für Staatssicherheit \\
        & Ehemaliger Inlands- und Auslandsgeheimdienst der DDR \\ \hline
  PC & Personal Computer \\ \hline
  Firefox & Webbrowser von Mozilla \\ \hline
  Google Chrome & Webbrowser von Google \\ \hline
  Internet Explorer & Webbrowser von Microsoft \\ \hline
  MTA & Mail Transport Agent \\ \hline
  MDA & Mail Delivery Agent \\ \hline
  POP3 & Post Office Protocol Version 3 \\ \hline
  IMAP & Internet Message Access Protocol \\ \hline
  GMX & E-Mail Provider \\ \hline
  Backup & Datensicherung \\ \hline
  PGP & Pretty Good Privacy \\ \hline
  GPG & GNU Privacy Guard \\ \hline
  IP & Internet Protocol \\ \hline
  IP Adresse & Eindeutige Identifikationsnummer eines Internetanschlusses \\ \hline
  DNS & Domain Name System \\
      & Dient der Auflösung von Internet Domains \\ \hline
  TOR & The Onion Routing \\
      & Netzwerk zur Anonymisierung von Verbindungsdaten \\ \hline
  HTTP & Hypertext Transfer Protocol \\
       & Protokol zur Übertragung von Daten über ein Netzwerk \\ \hline
  Linux & Unix Betriebssystem \\ \hline
  Raspberry Pi & kreditkartengrosser Einplatinencomputer \\ \hline
\end{tabular*}
