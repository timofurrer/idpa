\section{Cloud-Computing - drohende Gewitterwolke oder hilfreicher Schattenspender?}
Die Begriffe Cloud bzw. Cloud-Computing wurden in den letzten paar Jahren immer populärer. Das Interesse an der mystischen Wolke und deren Konzept ist stetig gewachsen und ein Ende ist nicht in Sicht. Gerade in den letzten paar Monaten, während den Enthüllungen Edward Snowdens über die Machenschaften diverser Geheimdienste, bekam die Cloud aber auch einen bitteren Beigeschmack. Für Privatanwender und Unternehmen gleichermassen stellt sich die Frage, ob ihre Geheimnisse bei Anbietern wie Microsoft, Google oder Amazone sicher sind. Um diese Frage überhaupt beantworten zu können, muss man das Konzept von Cloud-Computing verstehen.

\subsection{Was ist Cloud-Computing?}
Cloud-Computing bezeichnet das Modell, IT-spezfifische Dienste einem Kunden abstrahiert über ein Netzwerk zur Verfügung zu stellen und an dessen Bedarf anzupassen. Der Zugriff auf die Dienste soll dabei möglichst bequem und allgegenwärtig möglich sein. Die zur Verfügung gestellten Dienste können dabei von hardwareseitiger Natur sein (Rechenkapazitäten, Netzwerkkapazitäten, Rechenleistung) oder in Form von fertiger Software angeboten werden 
\footnote{\url{http://csrc.nist.gov/publications/nistpubs/800-145/SP800-145.pdf}} 
\footnote{\url{https://de.wikipedia.org/wiki/Cloud-Computing}}. 
Die zuvor erwähnte Abstraktion ist dabei ein sehr wichtiger Punkt bei Cloud-Computing. Privatpersonen oder Unternehmen die ein funktionierendes System oder eine funktionierend Software wünschen, ohne sich mit den technischen Aspekten auseinandersetzen zu müssen, kommen so voll auf ihre Kosten. Die Preise für die verschiedenen Dienste werden oft in Form von Abonnements angeboten und dabei zusätzlich an den Bedarf angepasst. Der Kunde erhält somit ein auf ihn zugeschnürtes Paket, das er nur noch nützen muss.
\\
\\
Das Konzept scheint aufzugehen und der Beweis dafür sind die unzähligen Firmen/Dienste, welche in den letzten Jahren, wie Pilze im Wald, aus dem Boden spriessen:
\begin{itemize}
\item SkyDrive (Online-Speicher, Microsoft)
\item Google's diverse Angebote wie Google Drive (Online-Speicher) und Google Docs (Office Suite, Equivalent zu Office 365 [siehe nächstes Kapitel])
\item Dropbox (Online-Speicher, Dropbox, Inc.)
\item Evernote (Verwalten von Notizen, Evernote Corporation)
\item ownCloud (Online-Speicher, ownCloud Inc. / Community)
\item u.v.m.
\end{itemize}

Die Wolke wächst kontinuierlich an, aber welche Risiken sind damit verbunden? Welchen Preis (von Geld einmal abgesehen) muss eine Privatperson oder ein Unternehmen bezahlen, um in den Genuss der Cloud zu kommen?

\subsubsection{Alltagsbeispiel – Microsoft Office}
Eines der wohl am meisten verwendeten Programme in der Geschäftswelt und auch privat ist die Office Suite von Microsoft. Diese wird nun, dem Trend folgend, seit ungefähr zwei Jahren ebenfalls über die Cloud angeboten. Die Vorteile dabei scheinen, zumindest auf den ersten Blick, ein Kaufgrund zu sein: 
\footnote{\url{https://office.microsoft.com/de-ch/products/microsoft-office-365-home-premium-kaufen-FX102853961.aspx}}

\begin{itemize}
\item Stets neueste Version der Programme
\item Auf bis zu 5 Geräten nutzen
\item Offline und online nutzbar --> Installation nicht zwingend erforderlich
\item 20 GB+ Online-Speicher, um Dokumente zu speichern und zu synchronisieren
\item Monatlich oder jährlich bezahlbar
\item 60+ „Skype-Minuten“ monatlich in alle Länder (Skype ist eine Telefonie-Software, welche über das Internet kommuniziert)
\end{itemize}

Das Ziel ist es natürlich, Office hauptsächlich in Verbindung mit der Cloud zu nutzen. Nur dann kann man von überall seine Dokumente bearbeiten oder neue erstellen, ohne dass man die Programme auf den jeweiligen Computern installiert haben muss. Die Daten landen dann alle in der Cloud, womit eigentlich die Server von Microsoft gemeint sind. Microsoft hat dafür einen eigenen Dienst namens „Sky-Drive“, der einfach zu nutzen und auf allen möglichen Geräten nutzbar sein soll . Ein Käufer von Office 365 erhält also kurzgesagt die volle Funktionalität, ohne sich viel mit technischen Fragen auseinandersetzen zu müssen.
Das Angebot richtet sich nicht nur an Privatpersonen, sondern auch an Firmen, wobei natürlich die Ausstattung und die Preise variieren.
\\
\\
Das obige Beispiel ist nur ein klitzekleiner Teil der riesigen Palette an angebotenen Diensten und Produkten. Auf den ersten Blick scheint es auch nichts Schlechtes daran zu geben und natürlich schreiben die jeweiligen Anbieter die mit ihren Diensten verbundenen Risiken nicht auf ihre Flaggen. Schaut man aber genauer hin, tauchen Fragen auf, die gerade mit der zurzeit laufenden Debatte um Datenschutz und Anonymität im Netz vieles überschatten, was zuvor rosig wirkte.

\subsection{Gewitterwolke statt Schattenspender?}
Das grosse Wort, das bei Debatten über die Risiken von Cloud Computing im Raum steht und immer öfter auch ausgesprochen wird, lautet Datenschutz. Datenschutz kann je nach Betrachtungsweise verschiedene Dinge ansprechen:
\footnote{\url{https://de.wikipedia.org/wiki/Datenschutz}}

\begin{itemize}
\item Schutz vor missbräuchlicher Datenverarbeitung
\item Schutz des Rechts auf informationelle Selbstbestimmung
\item Schutz des Persönlichkeitsrechts bei der Datenverarbeitung
\item Schutz der Privatsphäre
\end{itemize}

Einfach ausgedrückt, ist das Ziel von Datenschutz der Schutz aller persönlicher Daten eines jeden Bürgers. Jeder Bürger soll selbst entscheiden können, welche seiner persönlichen Daten wem zugänglich sind, zu welchem Zeitpunkt dies geschieht und in welcher Art. Die Frage lautet nun, ob Cloud-Dienste diesen Schutz gewährleisten können oder ob er allein in der Natur der Cloud-Dienste schon aufgehoben ist. Da die Anbieter und ihre Rechenzentren sich oftmals in einem anderen Land oder gar auf einem anderen Kontinent befinden als der Kunde, kann dieser kaum kontrollieren, was mit seinen Daten wirklich passiert. Die Anbieter müssen sich zudem meist an die Gesetze halten, die in dem Land gelten, in dem die Server stehen bzw. ihr Firmensitz sich befindet. Es kann also schnell einmal vorkommen, dass der Kunde und der Anbieter allein schon von der Gesetzesgrundlage her andere Vorstellungen haben. Den Kunden bleibt nichts anderes übrig, als entweder auf die Unternehmen zu vertrauen oder ganz auf die Dienste zu verzichten. Heute weiss man, dass es tatsächlich so ist, dass der Kunde nicht über alles in Kenntnis gesetzt wird, was mit seinen Daten passiert. Wie schon in einem vorherigen Kapitel erwähnt, fordern Geheimdienste immer wieder Daten direkt bei den Unternehmen an und verbieten diesen darüber hinaus, ihrerseits den Kunden darüber in Kenntnis zu setzen. 
Aber nicht nur Datenschutz ist im Zusammenhang mit Cloud-Computing wichtig. Die Informationssicherheit ist ebenfalls ein Punkt, der nicht vergessen werden sollte. Folgende Fragen verlangen dabei nach einer Antwort:
\footnote{\url{https://de.wikipedia.org/wiki/Informationssicherheit}}

Kann der Anbieter gewährleisten, dass

\begin{itemize}
\item Daten nicht von Dritten gelesen, gelöscht oder modifiziert werden?
\item Daten nicht unabsichtlich verändert werden und die Änderungen nachvollziehbar sind?
\item Daten nicht verloren gehen oder nicht abrufbar sind auf Grund eines systembedingten Ausfalls?
\end{itemize}

Hier geht es mehr um technische Fragen als um politische aber trotzdem muss es auf jede eine Antwort geben, wenn man seine Daten in guten (oder eben nicht guten) Händen wissen will.

Hat man sich erst einmal ein Bild über die aktuelle Lage gemacht und ist sich der Chancen und Risiken von Cloud-Computing bewusst, bleibt die Frage, wie man denn nun fortfahren soll. Total auf die Dienste verzichten wäre eine Verschwendung von hilfreicher Technologie und blindes Vertrauen mit grossen Risiken verbunden. Es gibt zum Glück verschiedene Lösungsansätze, die nicht ganz so radikal sind. Man könnte zum Beispiel nur Unternehmen beauftragen, die ihre Server im selben Land haben, womit man mehr Kontrolle über das Geschehen hätte. Das klappt aber nur, wenn sich Polizei und die Geheimdienste besagten Landes an Gesetze mit demokratischem Hintergrund halten müssen und dies nachweislich auch tun. Schlägt man diesen Weg ein, bleiben aber immer noch gewisse Zweifel bestehen, die man vielleicht nie wird aus dem Weg räumen können. Möchte man ganz sichergehen, bleibt einem Unternehmen nichts anderes Übrig, als eine eigene Cloud aufzubauen. Auf Firmenebene braucht man dafür die entsprechende Hardware, geschultes Personal und  Software, die die gewünschte Funktionalität zur Verfügung stellt. Die unmittelbaren Kosten können somit sehr gross sein. Die Frage ist, ob es sich auf Dauer für das Unternehmen lohnt. Privat könnte das ganze schon lukrativer sein. Man könnte beispielsweise unter Freunden und/oder Verwandten eine eigene Cloud einrichten und gemeinsam nutzen. Ein gewisses technisches Know-How ist aber auch hier gefragt. 

\subsection{ownCloud}
Unsere Versuchsreihe soll nun um einen weiteren Versucht erweitert werden. Wir wollen eine eigene Cloud aufbauen, die hauptsächlich dem Speichern und Synchronisieren von Daten dient. Das Ziel ist es, eine Lösung zu finden, die auch Laien verstehen und selbst umsetzten könnten, ohne grosses technisches Know-How anhäufen zu müssen. Für unseren Versuch verwenden wir eine Software namens „ownCloud“, welches richtig installiert und konfiguriert, die Cloud-Umgebung zur Verfügung stellen soll. 

\begin{quote}
ownCloud gives you universal access to your files through a web interface or WebDAV. It also provides a platform to easily view \& sync your contacts, calendars and bookmarks across all your devices and enables basic editing right on the web.
\footnote{\url{http://owncloud.org/about/}}	
\end{quote}

„ownCloud“ ist ein Programm, das einen ortsunabhängigen Speicherbereich zur Verfügung stellt, der über eine grafische Benutzeroberfläche verwaltet werden kann. Man kann Dateien verwalten, Kontakte \& Kalender synchronisieren und je nach Anwendung, direkt über den WebBrowser Files editieren. Das Tolle an „ownCloud“ ist - abgesehen von den bereits genannten Funktionen - dass es jeder kostenlos herunterladen und auf dem eigenen Server installieren kann. Einer eigenen Cloud für ein Unternehmen stünde somit nichts im Wege. Es unterstützt zudem alle gängigen Betriebssysteme und bietet einen separaten Desktop-Client, der auf den jeweiligen Rechnern oder gar dem Smartphone installiert werden kann, um Daten zu hochzuladen, runterzuladen oder zu synchronisieren.
Besonders erwähnenswert im Zusammenhang mit dem Thema Datenschutz sind folgende Punkte:

\begin{itemize}
\item Verschlüsselung der Daten auf dem Server
\item Verschlüsselte Übertragung per SSL/TLS
\item Open-Source-Software
\end{itemize}

„ownCloud“ ist ein Open-Source Programm. Das bedeutet, dass man neben dem ausführbaren Programm auch den Programmcode beziehn kann. Beides darf man nach belieben kopieren, weitergeben oder modifizieren. Gewisse Einschränkungen oder Regeln, an die man sich zu halten hat, hängen dabei von der jeweils verwendeten Lizenz ab, unter der das Programm veröffentlicht wurde.
\footnote{\url{http://opensource.org/osd}} 
\footnote{\url{https://de.wikipedia.org/wiki/Open_Source}} 
Die Eigenschaften von Open-Source-Software können je nach Anwendungsgebiet Fluch oder Segen sein. Einerseits kann dadurch jeder nachprüfen, was der Programmcode macht. Andererseits könnte aber genau diese Offenheit dazu genutzt werden, um mutmasslichen Schadcode einzuschleusen. Es hängt dann von der Community, einem selbst oder dem Zufall ab, ob solche Hintertüren entdeckt und geschlossen werden. Würde man wirklich auf Nummer sicher gehen wollen, bliebe einem nichts anderes übrig, als den ganzen Code zu kontrollieren, was je nach grösse des Programms unglaublich viel Zeit und somit auch Geld in Anspruch nehmen kann. Die Wahrscheinlichkeit des genannte Szenarios ist aber eher klein und sicherlich nicht grösser, als bei kommerzieller Software.

Im Endeffekt hat ownCloud das Ziel, eigens wartbar, kontrollierbar und frei (hier im Sinne von Open-Source) zu sein. Die versprochenen Funktionalitäten decken dabei schon fast alle Bedürfnisse des 0815-Anwenders ab und spreche somit für das Programm als Alternative zu den Grossen im Geschäft. Auf Details zu Funktionalitäten sowie Details zu Vor- und Nachteilen wird nachfolgend eingegangen.

\subsection{Wie funktioniert ownCloud}
Hierhin kommt ein Text, der grob die Funktionsweise von ownCloud beschreibt.

\subsection{Vor- und Nachteile}
Hier werden die Vor- und Nachteile von ownCloud aufgelistet, ohne gross auf sie einzugehen.

!! -- Lauftext, noch nicht final -- !!
Die Grösse des auf Computern verfügbaren Speichers hat über die Jahre exponentiell zugenommen. Disketten, auf die nicht einmal ein MP3-Song gepasst hätte, sind heute kaum noch in Verwendung. Viele Computer (PC, Notebook) kommen standardmässig mit über 100 GB Speicher ausgestattet daher. Platzmangel gibt es kaum. Möchte man den Überblick über seine Daten bewahren und diese auch vor Ausfällen und Hardware-Fehlern schützen, muss man sie hegen und pflegen. Backups (Abbilder der Daten) sind Pflicht, klare Ordnerstrukturen erwünscht. Das alles bedeutet aber viel Aufwand und setzt ebenfalls ein gewisses technisches Know-How voraus. Es ist doch ziemlich mühsam, Daten auf einen USB-Datenträger zu kopieren, nur um diese dann auf eine Zweitcomputer (zu Hause oder in der Schule) zu transferieren. Hier kommt die Cloud ins Spiel. Dienstleister bieten eine Sicherung und 24/7 Online-Zugriff zu Verfügung. Je nach Anbieter und Preis kommen noch weitere Dienste hinzu. 

Man gibt für den Gebrauch dieser Dienste aber die volle Kontrolle über seine Daten ab. Man muss den Anbietern vertrauen. Der Ottonormalverbraucher macht sich wohl nicht viele Gedanken über Risiken, Datenschutz und ähnliche Punkte. Er kann sich über ein Programm auf dem Computer oder dem Webbrowser mit ``seiner'' Cloud verbinden.

\subsection{Versuchsbeschreibung}
Hierhin kommt ein Text, der den Ablauf des Versuchs und die Ergebnisse beschreibt (inklusive Fazit?).

%Quellen Text:
%https://de.wikipedia.org/wiki/Cloud-Computing
%https://office.microsoft.com/de-ch/products/microsoft-office-365-home-premium-kaufen-FX102853961.aspx#SkyDriveStoragehttp://windows.microsoft.com/en-us/skydrive/compare
%https://de.wikipedia.org/wiki/Datenschutz
%https://de.wikipedia.org/wiki/Informationssicherheit
%https://de.wikipedia.org/wiki/OwnCloud
%http://opensource.org/osd
%https://de.wikipedia.org/wiki/Open_Source
%http://owncloud.org/about/

%Bilder:
