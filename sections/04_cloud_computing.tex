\newpage
\section{Cloud-Computing - drohende Gewitterwolke oder hilfreicher Schattenspender?}
Der Begriff Cloud bzw. Cloud-Computing wurden in den letzten paar Jahren immer populärer. Das Interesse an der mystischen Wolke und deren Konzept ist stetig gewachsen und ein Ende ist nicht in Sicht. Gerade in den letzten Monaten, während den Enthüllungen Edward Snowdens über die Machenschaften der diversen Geheimdienste, bekam die Cloud aber auch einen bitteren Beigeschmack. Für Privatanwender und Unternehmen gleichermassen stellt sich die Frage, ob ihre Geheimnisse bei Anbietern wie Microsoft, Google oder Amazon sicher sind. Um diese Frage überhaupt beantworten zu können, muss man das Konzept von Cloud-Computing verstehen.

\subsection{Was ist Cloud-Computing?}
Cloud-Computing bezeichnet das Modell, IT-spezfifische Dienste einem Kunden abstrahiert über ein Netzwerk zur Verfügung zu stellen und an dessen Bedarf anzupassen. Der Zugriff auf die Dienste soll dabei möglichst bequem und allgegenwärtig möglich sein. Die zur Verfügung gestellten Dienste können dabei von hardwareseitiger Natur sein (Rechenkapazitäten, Netzwerkkapazitäten, Rechenleistung) oder in Form von fertiger Software angeboten werden\footnote{\url{http://csrc.nist.gov/publications/nistpubs/800-145/SP800-145.pdf}} \footnote{\url{https://de.wikipedia.org/wiki/Cloud-Computing}}.
Die zuvor erwähnte Abstraktion ist dabei ein sehr wichtiger Punkt. Privatpersonen oder Unternehmen, die ein funktionierendes System oder eine funktionierende Software wünschen, ohne sich mit den technischen Aspekten auseinandersetzen zu müssen, kommen so voll auf ihre Kosten. Die Preise für die verschiedenen Dienste werden oft in Form von Abonnements angeboten und dabei zusätzlich an den Bedarf angepasst. Der Kunde erhält somit ein auf ihn zugeschnittenes Paket, das er nur noch einsetzen muss.
\\
\\
Das Konzept scheint aufzugehen und der Beweis dafür sind die unzähligen Firmen/Dienste, welche in den letzten Jahren, wie Pilze im Wald, aus dem Boden spriessen:
\begin{itemize}
\item SkyDrive (Online-Speicher, Microsoft)
\item Google's diverse Angebote wie Google Drive (Online-Speicher) und Google Docs (Office Suite mit beschränktem Funktionsumfang)
\item Dropbox (Online-Speicher, Dropbox, Inc.)
\item Evernote (Verwalten von Notizen, Evernote Corporation)
\item ownCloud (Online-Speicher, ownCloud Inc./Community)
\item u.v.m.
\end{itemize}

Die obige Aufzählung ist nur ein kleiner Ausschnitt des tatsächlichen Angebots. Die Wolke wächst kontinuierlich an. Aber welche Risiken sind damit verbunden? Welchen Preis - von Geld einmal abgesehen - muss eine Privatperson oder ein Unternehmen bezahlen, um in den Genuss der Cloud zu kommen?

\subsubsection{Alltagsbeispiel – Microsoft Office}
Eines der wohl am meisten verwendeten Programme in der Geschäftswelt und auch privat ist die Office Suite von Microsoft. Diese wird nun, dem Trend folgend, seit ungefähr zwei Jahren ebenfalls über die Cloud angeboten. Die Vorteile dabei scheinen, zumindest auf den ersten Blick, ein Kaufgrund zu sein:

\begin{itemize}
\item Stets neueste Version der Programme
\item Auf bis zu 5 Geräten nutzbar
\item Offline und online nutzbar, Installation nicht zwingend erforderlich
\item 20 GB+ Online-Speicher, um Dokumente zu speichern und zu synchronisieren
\item Monatlich oder jährlich bezahlbar
\item 60+ „Skype-Minuten“ monatlich in alle Länder (Skype ist eine Telefonie-Software, welche über das Internet kommuniziert)\footnote{\url{https://office.microsoft.com/de-ch/products/microsoft-office-365-home-premium-kaufen-FX102853961.aspx}}
\end{itemize}

Das Ziel ist es natürlich, Office hauptsächlich in Verbindung mit der Cloud zu nutzen. Nur dann kann man von überall seine Dokumente bearbeiten oder neue erstellen, ohne dass man die Programme auf den jeweiligen Computern installiert haben muss. Die Daten landen dann alle in der Cloud, womit eigentlich die Server von Microsoft gemeint sind. Microsoft hat dafür einen eigenen Dienst namens „Sky-Drive“, der einfach zu nutzen und auf allen möglichen Geräten nutzbar sein soll . Ein Käufer von Office 365 erhält also kurzgesagt die volle Funktionalität, ohne sich viel mit technischen Fragen auseinandersetzen zu müssen.
Das Angebot richtet sich nicht nur an Privatpersonen, sondern auch an Firmen, wobei natürlich die Ausstattung und die Preise variieren.
\\
\\
Das obige Beispiel ist nur ein klitzekleiner Teil der riesigen Palette an angebotenen Diensten und Produkten. Auf den ersten Blick scheint es auch nichts Schlechtes daran zu geben und natürlich schreiben die jeweiligen Anbieter die mit ihren Diensten verbundenen Risiken nicht auf ihre Flaggen. Schaut man aber genauer hin, tauchen Fragen auf, die gerade mit der zurzeit laufenden Debatte um Datenschutz und Anonymität im Netz vieles überschatten, was zuvor rosig wirkte.

\subsection{Gewitterwolke statt Schattenspender?}
Das grosse Wort, das bei Debatten über die Risiken von Cloud Computing im Raum steht und immer öfter auch ausgesprochen wird, lautet Datenschutz. Datenschutz kann je nach Betrachtungsweise verschiedene Dinge ansprechen:

\begin{itemize}
\item Schutz vor missbräuchlicher Datenverarbeitung
\item Schutz des Rechts auf informationelle Selbstbestimmung
\item Schutz des Persönlichkeitsrechts bei der Datenverarbeitung
\item Schutz der Privatsphäre\footnote{\url{https://de.wikipedia.org/wiki/Datenschutz}}
\end{itemize}

Einfach ausgedrückt, ist das Ziel von Datenschutz der Schutz aller persönlicher Daten eines jeden Bürgers. Jeder Bürger soll selbst entscheiden können, welche seiner persönlichen Daten wem zugänglich sind, zu welchem Zeitpunkt dies geschieht und in welcher Art. Die Frage lautet nun, ob Cloud-Dienste diesen Schutz gewährleisten können oder ob er allein in der Natur der Cloud-Dienste schon aufgehoben ist. Da die Anbieter und ihre Rechenzentren sich oftmals in einem anderen Land oder gar auf einem anderen Kontinent befinden als der Kunde, kann dieser kaum kontrollieren, was mit seinen Daten wirklich passiert. Die Anbieter müssen sich zudem meist an die Gesetze halten, die in dem Land gelten, in dem die Server stehen bzw. sich ihr Firmensitz befindet. Es kann also schnell einmal vorkommen, dass der Kunde und der Anbieter allein schon von der Gesetzesgrundlage her andere Vorstellungen haben. Den Kunden bleibt nichts anderes übrig, als entweder auf die Unternehmen zu vertrauen oder ganz auf die Dienste zu verzichten. Heute weiss man, dass es tatsächlich so ist, dass der Kunde nicht über alles in Kenntnis gesetzt wird, was mit seinen Daten passiert. Wie schon in einem vorherigen Kapitel erwähnt, fordern Geheimdienste immer wieder Daten direkt bei den Unternehmen an und verbieten diesen darüber hinaus, ihrerseits den Kunden darüber in Kenntnis zu setzen.
Aber nicht nur Datenschutz ist im Zusammenhang mit Cloud-Computing wichtig. Die Informationssicherheit ist ebenfalls ein Punkt, der nicht vergessen werden sollte. Folgende Fragen verlangen dabei nach einer Antwort:
\\
\\
Kann der Anbieter gewährleisten, dass

\begin{itemize}
\item Daten nicht von Dritten gelesen, gelöscht oder modifiziert werden?
\item Daten nicht unabsichtlich verändert werden und getätigte Änderungen nachvollziehbar sind?
\item Daten nicht verloren gehen oder nicht abrufbar sind auf Grund eines systembedingten Ausfalls?\footnote{\url{https://de.wikipedia.org/wiki/Informationssicherheit}}
\end{itemize}

Hier geht es mehr um technische Fragen als um politische aber trotzdem muss es auf jede eine Antwort geben, wenn man seine Daten in guten (oder eben nicht guten) Händen wissen will.

\subsection{Was man tun kann - Nutzen und Aufwand}
Hat man sich erst einmal ein Bild über die aktuelle Lage gemacht und ist sich der Chancen und Risiken von Cloud-Computing bewusst, bleibt die Frage, wie man denn nun fortfahren soll. Total auf die Dienste zu verzichten wäre eine Verschwendung von hilfreicher Technologie. Blindes Vertrauen hingegen wäre mit grossen Risiken verbunden. Es gibt zum Glück verschiedene Lösungsansätze, die nicht ganz so radikal sind. Man könnte zum Beispiel nur Unternehmen beauftragen, die ihre Server im selben Land haben, womit man mehr Kontrolle über das Geschehen hätte. Das klappt aber nur, wenn sich Polizei und die Geheimdienste besagten Landes an die jeweiligen Gesetze halten müssen und dies nachweislich auch tun. Schlägt man diesen Weg ein, bleiben aber immer noch gewisse Zweifel bestehen, die man vielleicht nie wird aus dem Weg räumen können. Möchte man ganz sichergehen, bleibt einem Unternehmen nichts anderes übrig, als eine eigene Cloud einzurichten. Auf Firmenebene braucht man dafür die entsprechende Hardware, geschultes Personal und  Software, die die gewünschte Funktionalität zur Verfügung stellt. Die unmittelbaren Kosten können somit stark in die Höhe schnellen. Auf Dauer aber könnte sich die Investition allemal lohnen. Bei der privaten Nutzung verhält es sich ähnlich. Die Hardware muss aber in diesem Einsatzgebiet deutlich weniger leistungsfähig sein. Zudem könnte man zusammen mit Freunden und/oder Verwandten eine gemeinsame Cloud einrichten. Ein gewisses technisches Know-How ist aber auch hier gefragt.

\subsection{Eigene Cloud einrichten - leicht gemacht}
Die Idee der eigenen Cloud soll nun in die Praxis umgesetzt werden. Es soll eine private Cloud eingerichtet werden, die hauptsächlich dem Speichern und Synchronisieren von Daten dient. Das Ziel ist es, eine Lösung zu finden, die auch Laien verstehen und selbst umsetzen könnten, ohne grosses technisches Know-How anhäufen zu müssen. In den folgenden Kapiteln wird zuerst näher auf die im Versuch eingesetzte Software eingegangen. Anschliessend wird dann ein stark zusammengefasster Abschnitt Klarheit darüber geben, ob und wie leicht ein solches Projekt umzusetzen ist.

\subsubsection{ownCloud}
Für den Versuch wird eine Software namens ``ownCloud'' verwendet. ownCloud ist ein Programm, das einen ortsunabhängigen Speicherbereich zur Verfügung stellt, der über eine grafische Benutzeroberfläche verwaltet werden kann. Man kann Dateien verwalten, Kontakte \& Kalender synchronisieren und je nach Anwendung, direkt über den Webbrowser Dateien editieren\footnote{\url{http://owncloud.org/about/}}.
Das Tolle an ownCloud ist - abgesehen von den bereits genannten Funktionen - dass es jeder kostenlos herunterladen und auf dem eigenen Server installieren kann. Einer eigenen Cloud für ein Unternehmen stünde somit nichts im Wege. Es unterstützt zudem alle gängigen Betriebssysteme und bietet dazu einen separaten Client, der auf den jeweiligen Rechnern oder gar dem Smartphone installiert werden kann, um Daten noch bequemer hochladen, herunterladen oder synchronisieren zu können.
Besonders erwähnenswert im Zusammenhang mit dem Thema Datenschutz sind folgende Punkte:

\begin{itemize}
\item Verschlüsselung der Daten auf dem Server
\item Verschlüsselte Übertragung (SSL/TLS)
\item Open-Source-Software
\end{itemize}

Wie oben bereits erwähnt ist ownCloud ist ein Open-Source-Programm. Das bedeutet, dass man neben dem ausführbaren Programm auch den Programmcode beziehen kann. Beides darf man nach Belieben kopieren, weitergeben oder modifizieren. Gewisse Einschränkungen oder Regeln, an die man sich zu halten hat, hängen dabei von der jeweils verwendeten Lizenz ab, unter der das Programm veröffentlicht wurde.\footnote{\url{http://opensource.org/osd}} \footnote{\url{https://de.wikipedia.org/wiki/Open_Source}}
Die Eigenschaften von Open-Source-Software können je nach Anwendungsgebiet Fluch oder Segen sein. Einerseits kann dadurch jeder nachprüfen, was der Programmcode macht. Andererseits könnte aber genau diese Offenheit dazu genutzt werden, um mutmasslichen Schadcode einzuschleusen. Es hängt dann von der Community, einem selbst oder dem Zufall ab, ob solche Hintertüren entdeckt und geschlossen werden. Würde man wirklich auf Nummer sicher gehen wollen, bliebe einem nichts anderes übrig, als den ganzen Code zu kontrollieren, was je nach Grösse des Programms unglaublich viel Zeit und somit auch Geld in Anspruch nehmen kann. Die Wahrscheinlichkeit des genannten Szenarios ist aber eher gering und sicherlich nicht grösser, als bei kommerzieller Software.
\\
\\
Im Endeffekt hat ownCloud das Ziel, wartbar, kontrollierbar und frei (hier im Sinne von Open-Source) zu sein. Die versprochenen Funktionalitäten decken dabei viele Bedürfnisse des 0815-Anwenders ab und sprechen somit für das Programm als Alternative zu den Grossen im Geschäft. Auf Details zu Funktionalitäten sowie Details zu Vor- und Nachteilen wird später noch genauer eingegangen.

\subsubsection{Wie funktioniert ownCloud}
ownCloud muss wie jedes andere Programm auf einem Computer installiert werden. Der Einsatz der Cloud ist hierbei massgeblich für die Entscheidung, welche Hardware zum Einsatz kommt. Braucht man die Cloud nur für sich selbst, kann ein alter PC dienen, der anderweitig nicht mehr gebraucht wird. Soll aber eine Cloud für ein kleines Unternehmen, die Familie oder einen Verein aufgesetzt werden, macht es sicherlich mehr Sinn, aktuelle und performante Hardware zu verwenden. Natürlich braucht es neben der Cloud-Software selbst ein Betriebssystem, um die Anwendung überhaupt laufen zu lassen. Dazu kommen noch andere Software-Komponenten, ohne die ``ownCloud" nicht sinnvoll genutzt werden kann.
\\
\\
Die Benutzung der Cloud bzw. der diversen Dienste ist ziemlich einfach. Jeder Benutzer kann mittels Webbrowser über ein sogenanntes ``Webinterface'' auf die Cloud zugreifen. Er kann beispielsweise Dokumente hoch- und herunterladen, Kontakte verwalten, Musik abspielen und vieles mehr.
Inzwischen gibt es sogar Programme für Android- und iOS-basierte Smartphones/Tabletcomputer, mit denen man unterwegs einfach auf die in der Cloud verwalteten Daten zugreifen kann.

\subsubsection{Vor- und Nachteile}
Wie schon erwähnt hat ownCloud einige Vorteile. Die Nachteile sollten aber auch nicht ausser Acht gelassen werden. Grundsätzlich ist man bei ownCloud genau richtig, möchte man Herr seiner Daten sein. Was dadurch aber nicht vermieden werden kann, ist eine gewisse Einarbeitungszeit bzw. ein gewisses technisches Know-How. Zusätzlich darf man auch nicht vergessen, dass die Cloud gewartet werden muss. Privatsphäre und Sicherheit haben ihren Preis und jede Privatperson bzw. jedes Unternehmen muss selbst entscheiden, was Priorität hat. Die Cloud ist letztendlich nur so sicher, wie der Anwender sie macht und selbst dann gibt es noch gewisse technologisch bedingte Einschränkungen.
Nachfolgend werden noch einmal Vor- und Nachteile stichwortartig aufgelistet, bevor der Versuch selbst näher beschrieben wird.
\\
\\
Vorteile
\begin{itemize}
\item Anwender ist der Souverän
\item Einsatz von Verschlüsselung bei Speicher und Übertragung
\item kostenlos
\item Support durch Community
\item Open-Source
\item "Leisetreter", man ist privater unterwegs
\end{itemize}

Nachteile:
\begin{itemize}
\item setzt gewisses technisches Know-How voraus
\item Wartung
\item Kauf eigener Hardware
\item Verbindungsgeschwindigkeit u.U. nicht so hoch wie bei grossen Anbietern
\end{itemize}

\subsubsection{Versuchsbeschreibung}
Dieses Kapitel schildert die Erfahrungen des Versuches, eine eigene Cloud mittels ownCloud einzurichten. Zum Schluss folgt ein Fazit und weitere Gedanken.
\\
Das Ziel war es, eine Cloud für das heimische Netzwerk einzurichten. Das bedeutet, dass man von zu Hause aus auf die Cloud und die angebotenen Dienste zugreifen kann. Ausserhalb des heimischen Netzes ist dies dann aber nicht mehr möglich. ownCloud kann aber mit ein bisschen Mehraufwand so konfiguriert werden, dass auch dies möglich ist. Will man die eigene Cloud professionell nutzen, sollte dieser Mehraufwand definitiv betrieben werden.
\\
Die Installation selbst ging erstaunlich einfach vonstatten. Es gibt in den weiten des Internets viele Anleitungen und Tipps, die einem die Arbeit deutlich erleichtern. Mit ein bisschen Fleiss und Geduld hat man das nötige Know-How so sehr schnell zusammen. Zuerst mussten die Hardware- und die Softwarekomponenten ausgewählt werden. In beiden Bereichen hat man einen gewissen Spielraum. Dann ging es auch schon los mit dem Aufsetzen des Betriebssystems. Einmal korrekt eingerichtet, konnten die Software-Komponenten installiert werden, die neben der Cloud-Software für eine korrekte Ausführung benötigt werden. ownCloud selbst benötigte dann den kleinsten Aufwand. Innerhalb kürzester Zeit ist das Programm installiert wie auch eingerichtet und kurze Zeit später findet man sich schon auf der Hauptseite der Cloud wieder.

Standardmässig kann man sich mittels Webbrowser mit der Cloud verbinden. Die Bedienung geht dabei kinderleicht vonstatten. Es können mit wenigen Klicks alle Einstellungen vorgenommen und zusätzlich Funktionalitäten installiert werden. 

%Quellen Text:
%https://de.wikipedia.org/wiki/Cloud-Computing
%https://office.microsoft.com/de-ch/products/microsoft-office-365-home-premium-kaufen-FX102853961.aspx#SkyDriveStoragehttp://windows.microsoft.com/en-us/skydrive/compare
%https://de.wikipedia.org/wiki/Datenschutz
%https://de.wikipedia.org/wiki/Informationssicherheit
%https://de.wikipedia.org/wiki/OwnCloud
%http://opensource.org/osd
%https://de.wikipedia.org/wiki/Open_Source
%http://owncloud.org/about/

%Bilder:
