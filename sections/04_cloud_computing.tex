\section{Cloud-Computing - drohende Gewitterwolke oder hilfreicher Schattenspender?}

\subsection{Was ist Cloud-Computing?}
Die Begriffe Cloud bzw. Cloud-Computing wurden in den letzten paar Jahren immer populärer. Das Interesse an der mystischen Wolke und deren Konzept ist stetig gewachsen und ein Ende ist nicht in Sicht. Gerade in den letzten paar Monaten, während den Enthüllungen Edward Snowdens über die Machenschaften diverser Geheimdienste, bekam die Cloud aber auch einen bitteren Beigeschmack. Für Privatanwender und Unternehmen gleichermassen stellt sich die Frage, ob ihre Geheimnisse bei Anbietern wie Microsoft, Google oder Amazone sicher sind. Um diese Frage überhaupt beantworten zu können, muss man das Konzept von Cloud-Computing verstehen.
\\
\\
Cloud-Computing bezeichnet die Idee, IT-spezfifische Dienste einem Kunden abstrahiert über ein Netzwerk zur Verfügung zu stellen und an dessen Bedarf anzupassen. Mit abstrahiert ist gemeint, dass der Kunde sich dabei nicht mit den technischen Einzelheiten auseinandersetzen muss. Die zur Verfügung gestellten Dienste können dabei von hardwareseitiger Natur sein (Rechenkapazitäten, Netzwerkkapazitäten, Rechenleistung) oder in Form von fertiger Software angeboten werden. Die zuvor erwähnte Abstraktion ist dabei ein sehr wichtiger Punkt bei Cloud-Computing. Privatpersonen oder Unternehmen die ein funktionierendes System oder eine funktionierend Software wünschen, ohne sich mit den technischen Aspekten auseinandersetzen zu müssen, kommen so voll auf ihre Kosten. Die Preise für die verschiedenen Dienste werden oft in Form von Abonnements angeboten und dabei zusätzlich an den Bedarf angepasst. Der Kunde erhält somit ein auf ihn zugeschnürtes Paket, das er nur noch nützen muss.
\\
\\
Das Konzept scheint aufzugehen und der Beweis dafür sind die unzähligen Firmen/Dienste, welche in den letzten Jahren, wie Pilze im Wald, aus dem Boden spriessen:
\begin{itemize}
\item SkyDrive (Online-Speicher, Microsoft)
\item Google's diverse Angebote wie Google Drive (Online-Speicher) und Google Docs (Office Suite, Equivalent zu Office 365 [siehe nächstes Kapitel])
\item Dropbox (Online-Speicher, Dropbox, Inc.)
\item Evernote (Verwalten von Notizen, Evernote Corporation)
\item ownCloud (Online-Speicher, ownCloud Inc. / Community)
\end{itemize}

Die Wolke wächst kontinuierlich an, aber welche Risiken sind damit verbunden? Welchen Preis (von Geld einmal abgesehen) muss eine Privatperson oder ein Unternehmen bezahlen, um in den Genuss der Cloud zu kommen?

\subsection{Alltagsbeispiel – Microsoft Office}
Eines der wohl am meisten verwendeten Programme ist die Office Suite von Microsoft. Diese wird nun, dem Trend folgend, seit ungefähr zwei Jahren ebenfalls über die Cloud angeboten. Die Vorteile dabei scheinen, zumindest auf den ersten Blick, ein Kaufgrund zu sein:
\begin{itemize}
\item Stets neueste Version der Programme
\item Auf bis zu 5 Geräten nutzen
\item Offline und online nutzbar --> Installation nicht zwingend erforderlich
\item 20 GB+ Online-Speicher, um Dokumente zu speichern und zu synchronisieren
\item Monatlich oder jährlich bezahlbar
\item 60+ „Skype-Minuten“ monatlich in alle Länder
\end{itemize}

Das Ziel ist es natürlich, Office hauptsächlich in Verbindung mit der Cloud zu nutzen. Nur dann kann man von überall seine Dokumente bearbeiten oder neue erstellen, ohne dass man die Programme auf den jeweiligen Computern installiert haben muss. Die Daten landen dann alle in der Cloud. Microsoft hat dafür einen eigenen Dienst namens „Sky-Drive“, der einfach zu nutzen und auf allen möglichen Geräten nutzbar sein soll . Ein Käufer von Office 365 erhält also kurzgesagt die volle Funktionalität, ohne sich viel mit technischen Fragen auseinandersetzen zu müssen.
Das Angebot richtet sich nicht nur an Privatpersonen, sondern auch an Firmen, wobei die Ausstattung und die Preise variieren.
\\
\\
Das obige Beispiel ist nur ein klitzekleiner Teil der riesigen Palette an angebotenen Diensten und Produkten. Auf den ersten Blick scheint es auch nichts Schlechtes daran zu geben und natürlich schreiben die jeweiligen Anbieter die mit ihren Diensten verbundenen Risiken nicht auf ihre Flaggen. Schaut man aber genauer hin, tauchen Fragen auf, die gerade mit der zurzeit laufenden Debatte um Datenschutz und Anonymität im Netz vieles überschatten, was zuvor rosig wirkte.

\subsubsection{Gewitterwolke statt Schattenspender?}
Das grosse Wort, das im Raum steht und immer öfter auch ausgesprochen wird, lautet Datenschutz. Datenschutz kann je nach Betrachtungsweise verschiedene Dinge ansprechen:
\begin{itemize}
\item Schutz vor missbräuchlicher Datenverarbeitung
\item Schutz des Rechts auf informationelle Selbstbestimmung
\item Schutz des Persönlichkeitsrechts bei der Datenverarbeitung
\item Schutz der Privatsphäre
\end{itemize}

Einfach ausgedrückt, ist das Ziel von Datenschutz der Schutz aller persönlicher Daten eines jeden Bürgers. Jeder Bürger soll selbst entscheiden können, welche seiner persönlichen Daten wem zugänglich sind, zu welchem Zeitpunkt dies geschieht und in welcher Art. Die Frage lautet nun, ob Cloud-Dienste diesen Schutz gewährleisten können oder ob er allein in der Natur der Cloud-Dienste schon aufgehoben ist. Da die Anbieter und ihre Rechenzentren sich oftmals in einem anderen Land oder gar auf einem anderen Kontinent befinden als der Kunde, kann dieser kaum kontrollieren, was mit seinen Daten wirklich passiert. Ihm bleibt nichts andere übrig, als entweder auf die Unternehmen zu vertrauen oder ganz auf die Dienste zu verzichten. Heute weiss man, dass es tatsächlich so ist, dass der Kunde nicht über alles in Kenntnis gesetzt wird, was mit seinen Daten passiert. Geheimdienste fordern immer wieder Daten direkt bei den Unternehmen und verbieten diesen, ihrerseits den Kunden darüber in Kenntnis zu setzen. Aber nicht nur Datenschutz ist im Zusammenhang mit Cloud-Computing wichtig. Die Informationssicherheit ist es genauso.
Kann der Anbieter gewährleisten, dass:
\begin{itemize}
\item Daten nicht von Dritten gelesen, gelöscht oder modifiziert werden?
\item Daten nicht unabsichtlich verändert werden und die Änderungen nachvollziehbar sind?
\item Daten nicht verloren gehen oder nicht abrufbar sind auf Grund eines systembedingten Ausfalls?
\end{itemize}
Hier geht es mehr um technische Fragen als um politische aber trotzdem muss es auf jede eine Antwort geben, wenn man seine Daten in guten (oder eben nicht guten) Händen wissen will.

Total auf die Dienste verzichten wäre eine Verschwendung von hilfreicher Technologie und blindes Vertrauen mit grossen Risiken verbunden. Es gibt zum Glück verschiedene Lösungsansätze, die nicht ganz so radikal sind. Man könnte zum Beispiel nur Unternehmen beauftragen, die ihre Server im selben Land haben, womit man mehr Kontrolle über das Geschehen hätte. Das klappt aber nur, wenn sich Polizei und die Geheimdienste besagten Landes an Gesetze mit demokratischem Hintergrund halten müssen und dies nachweislich auch tun. Schlägt man diesen Weg ein, bleiben aber immer noch gewisse Zweifel bestehen, die man vielleicht nie wird aus dem Weg räumen können. Möchte man ganz sichergehen, bleibt einem Unternehmen nichts anderes Übrig, als eine eigene Cloud aufzubauen. Dafür braucht man die entsprechende Hardware, geschultes Personal und   Software, die die gewünschte Funktionalität zur Verfügung stellt. Die unmittelbaren Kosten können somit sehr gross sein. Die Frage ist, ob es sich auf Dauer für das Unternehmen lohnt.

\subsubsection{ownCloud}
Unsere Versuchsreihe soll nun um einen weiteren Versucht erweitert werden. Wir wollen eine eigene Cloud aufbauen, die hauptsächlich dem Speichern und Synchronisieren von Daten dient. Dazu verwenden wir eine Software namens „ownCloud“
\\
\\
„ownCloud“ ist ein Programm, das einen ortsunabhängigen Speicherbereich zur Verfügung stellt, der über eine grafische Benutzeroberfläche verwaltet werden kann. Das Tolle an „ownCloud“ ist, dass es jeder kostenlos herunterladen und auf dem eigenen Server installieren kann. Einer eigenen Cloud für ein Unternehmen stünde somit nichts im Wege. Es unterstützt zudem alle gängigen Betriebssysteme und bietet einen separaten Desktop-Client, der auf den jeweiligen Rechnern oder gar dem Smartphone installiert werden kann, um Daten zu hochzuladen, runterzuladen oder zu synchronisieren.
Besonders erwähnenswert im Zusammenhang mit dem Thema Datenschutz sind folgende Punkte:
\begin{itemize}
\item Verschlüsselung der Daten auf dem Server
\item Verschlüsselte Übertragung per SSL/TLS
\end{itemize}

Vorteile: / Nachteile:
\\....
\\....
\\
\\
„ownCloud“ ist ein open-source Programm. Das bedeutet, der Programmcode ist offen einsehbar, und ein jeder darf daran mitentwickeln. Das kann sowohl Fluch als auch Segen sein. Versteckte Hintertüren im Programm können durch Kontrolle ausgeschlossen werden aber wer sagt, dass nicht auch ein jeder Hintertüren einbauen kann? Möchte man eine sichere Lösung, bliebe einem also nichts anderes übrig, als den ganzen Code zu kontrollieren, was unglaublich viel Zeit und somit auch Geld in Anspruch nehmen kann. Wenn man die Daten zusätzlich auf den Servern verschlüsselt ablegt und jegliche Verbindungen ebenfalls verschlüsselt aufbaut, hat man schon deutlich mehr Kontrolle darüber, was mit den Daten geschieht. Vertrauen ist natürlich auch dann noch gefragt, aber das Ganze ist schon deutlich besser überblick- und kontrollierbar.
\\
\\
Quellen Text:
\\https://de.wikipedia.org/wiki/Cloud-Computing
%https://office.microsoft.com/de-ch/products/microsoft-office-365-home-premium-kaufen-FX102853961.aspx#SkyDriveStoragehttp://windows.microsoft.com/en-us/skydrive/compare
\\https://de.wikipedia.org/wiki/Datenschutz
\\https://de.wikipedia.org/wiki/Informationssicherheit
\\https://de.wikipedia.org/wiki/OwnCloud
\\
\\
Bilder:
%https://encrypted.google.com/search?tbm=isch&q=cloud
%20computng&tbs=imgo:1#q=cloud+computing&tbm=isch&tbs=imgo:1&facrc=_&imgdii=_&imgrc=FaydyLVFqECUYM%3A%3BLDF7GxZ2-uu6tM%3Bhttp%253A%252F%252Ftechosta.com%252Fwp-content%252Fuploads%252F2013%252F07%252Fclear-cloud-computing-diagram.jpg%3Bhttp%253A%252F%252Ftechosta.com%252Fbenefits-of-cloud-computing%252F%3B3960%3B2970
