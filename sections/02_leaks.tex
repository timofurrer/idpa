\newpage
\section{Edward Snowden Leaks}
Die Masse der im Laufe der Snowden-Enthüllungen veröffentlichten Zeitungsartikel, Newsartikel, Grafiken, Videos und sonstigen Beiträge im Netz ist enorm. Vor allem auch IT-Seiten wie beispielsweise ``heise online'', ein Nachrichtenticker mit Schwerpunkt Informations- und Telekommunikationstechnik, hat sich in den letzten Monaten intensiv dem Thema gewidmet. Wir möchten in diesem Abschnitt aber nur ein Teil dessen abdecken, was es zu wissen gibt. Bitte seien Sie, lieber Leser, ermutigt, sich tiefer mit der Thematik zu befassen, sollte Ihr Interesse geweckt sein. Sie werden nicht glauben, was man in den letzten Monaten so alles erfahren hat.  

\subsection{Was meinen die Volksvertreter dazu?}
Die Politiker und Regierungsvertreter von betroffenen Ländern, darunter Deutschland, Frankreich und Brasilien, haben in Bezug auf die unzähligen Enthüllunen, die seit Juni 2013 veröffentlicht wurden, die unterschiedlichsten Reaktionen gezeigt. Wenig überraschend ist es, dass zu Beginn alle schwer empört waren und Stellungnahmen seitens der Verantwortlichen verlangten. Und genauso schnell erklärte man die ganze Affäre für geklärt und versuchte, die aufgebrachte Meute zu beschwichtigen. Das geht aber nicht ganz so einfach, denn die konstante Flut an neuen Enthüllungen lässt sich nicht stoppen. Die amerikanische Regierung konnte offiziell bis Dezember 2013 nicht ermittlen, wie gross das Datenleck ist, das Edward Snowden verursacht hat.
\footnote{\url{http://www.spiegel.de/netzwelt/netzpolitik/nsa-raetselt-ueber-ausmass-der-snowden-enthuellungen-a-939145.html}} 
Das führt dazu, dass sich die Politiker, die echte Besorgnis gezeigt haben, in ihrer Meinung unterstützt finden und sich die ``Schönredner'' und Geheimdienstfreunde zwangsläufig mit der Problematik auseinandersetzen müssen.
Hinter Politik stecken immer auch Interessen. Man darf sich deshalb nicht zu sehr auf Aussagen von Schlüsselfiguren in der Weltpolitik verlassen, ganz egal, ob sie sich für oder gegen die Geheimdiesnte aussprechen. Viel mehr sollten die resultierenden Massnahmen als Mass genommen werden.

\subsection{Reaktionen der Bürger}
Die Volksvertreter sind die Vertreter des Volkes. Viel wichtiger als die Meinungen der Politiker, sind demnach die Meinungen des Bürger. Wenn wir hier von Bürgern sprechen, beziehen wir uns nicht auf ein spezifisches Land. Viel mehr meinen wir damit alle in den betroffenen Ländern lebenden Menschen. Wie nicht anders zu erwarten, gibt es auch hier Befürworter und Gegner. Aus Umfragen hat man herauslesen können, dass es viele gibt, die ihre Privatsphäre ohne grosse Bedenken hergeben, solange sie dafür Sicherheit erhalten. Auch fühlen sich viele gar nicht betroffen und glauben, dass ihnen durch die massenhafte Abhörung keine Nachteile entstehen. 
\footnote{\url{http://www.faz.net/aktuell/politik/bundestagswahl/neue-allensbach-analyse-wirkungslose-aufregung-12539865.html}}
\\
Man darf nicht vergessen, welche Faktoren bei solchen Überlegungen alle eine Rolle spielen. Viele wollen schlicht nicht auf die unzähligen Dienstleistungen verzichten, die mit grosser Warscheinlichkeit betroffen sind. Die heutige Kombination von sozialen Medien, ständig kommunizierenden Smartphones und Unterhaltung ist schliesslich ein immer festerer Bestandteil unseres Lebens. Nur ungern möchte man einen grösseren Teil des Komforts und die Möglichkeiten hingeben, um mehr Privatsphäre zu haben. Dazu kommt noch, dass man ja nicht mitkriegt, wenn man überwacht wird und oft denkt der Durchschnittsmensch, dass er ja wohl kaum von Interesse ist für die Geheimdienste dieser Welt. Und falls doch, dann hat er ja nichts zu verbergen.
\\
Es gibt aber auch viele Menschen, die sich in Folge der Enthüllungen äusserst besorgt zeigten und auf die Strasse gingen. Es gab Demonstrationen, unzählige Diskussionen in Foren und auf Newsseiten und andere Protestaktionen. Vor allem aus der IT-Welt hagelt es Kritik gegen die Machenschaften der Geheimdienste. Auf welchen Ebenen die Problematik angegangen wird, wird im nachfolgenden Abschnitt näher erläutert. 
