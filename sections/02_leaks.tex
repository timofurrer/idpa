\newpage
\section{Edward Snowden Leaks}
Die Masse der im Laufe der Snowden-Enthüllungen veröffentlichten Zeitungsartikel, Newsartikel, Grafiken, Videos und sonstigen Beiträge im Netz ist enorm.
Vor allem auch IT-Seiten wie beispielsweise ``heise online'', ein Nachrichtenticker mit Schwerpunkt Informations- und Telekommunikationstechnik, hat sich in den letzten Monaten intensiv dem Thema gewidmet.
Wir möchten in diesem Abschnitt aber nur einen Teil dessen abdecken, was es zu wissen gibt. Bitte seien Sie, lieber Leser, ermutigt, sich tiefer mit der Thematik zu befassen, sollte Ihr Interesse geweckt sein.
Sie werden nicht glauben, was man in den letzten Monaten so alles erfahren hat...

\subsection{Die Enthüllungen}
Eine Menge namhafter Firmen mussten zusehen wie teils, für sie, sehr unangenehme Wahrheiten ans Licht gebracht wurden. Wahrheiten, die selbst treue Kunden und Benutzer nachgedenklich gemacht haben: Soll ich denn wirklich noch meine E-Mails bei Google haben?, Kann ich noch ohne ausspioniert zu werden im Internet surfen?. Zurecht darf man sich diese Fragen stellen, denn es ist wichtig selbst grossen Firmen gegenüber kritisch zu sein.

\subsubsection{Google, Yahoo, Microsoft \& Co. - Kunden werden belauscht}
Noch vor ein paar Monaten dachte wohl fast jeder der ein E-Mail konnte bei Google, Yahoo, Microsoft oder Co hatte, dass diese nur für ihn selbst einsichtbar wären. Denn man hat natürlich ein besonders sicheres Passwort gewählt, dass auch den Vorgaben des entsprechenden Providers entsprach. Nach all den Veröffentlichungen von Edward Snowden hat man gemerkt, dass man selbst mit einem äusserst sicheren Passwort keines Wegs sicher ist. Der Provider hat die Kontrolle über die Server auf denen z.B. die Mails gespeichert werden. Verkauft der Provider diese Daten kann es für die Benutzer recht unangenehm werden. Die Privatspähre wird missbraucht und man hat praktisch keine Chance dem zu entgehen.
Aber selbst wenn der Provider dies nicht tut, so besteht dennoch die Gefahr dass Geheimdienste, wie die NSA, oder andere Organisationen beginnen sich in solche Dienste einzuhacken oder auf sonstige Art an die Daten zu gelangen.
Dies war beispielsweise der Fall mit dem Ausspäh-Werkzeut ``Muscular'' der NSA. Mit diesem Werkzeug war es der NSA möglich sich in die Glasfasernleitungen von Google einzuklinken und Daten Hunderter Millionen Nutzerkonten abzugreifen. Nicht nur Google war davon betroffen. Auch den Leitungen von Yahoo wurden private Nutzerdaten entzogen. Laut den NSA-Papieren vom 9. Januar 2013, schreiben die Zeitungen, hat die NSA Behörde innert eines Monates über 181 Millionen Datensätze von Google und Yahoo an Datenzentren des NSA-Hauptquartiers geschickt. Dazu muss noch gesagt werden, dass den Informationen zu folge, sich die NSA für dieses Werkzeug mit dem britischen Geheimdienst ``GCHQ'' zusammen getan hat.
Natürlich wurde dies von NSA-Chef General Keith B. Alexander vehement abgestritten: ``Wir haben keinen Zugang zu Google-Servern, Yahoo-Servern und so weiter.'' \footnote{\url{http://www.n-tv.de/politik/Google-kritisiert-Ausspaehung-durch-NSA-Gmail-Docs-und-Maps-offenbar-betroffen-article11640736.html}} \\ \\

Doch nicht nur E-Mail Konten waren betroffen. Laut einem Artikel des britischen Guardian hat die NSA in Zusammenarbeit mit dem GCHQ ein Programm namens ``Dishfire'' entwickelt, dass ``so ziemlich alles sammelt, was es findet''.
Informationen werden wahllos aus Reiseplänen, Adressbüchern, Finanztransaktionen und so weiter zusammengetragen und ausgewertet. Das Dokument vom Jahr 2011, auf dass sich der Artikel bezieht, macht deutlich, dass pro Tag zum Teil bis zu 200 Millionen SMS gesammelt werden konnten. Diese SMS waren Verknüpft mit dem Sender und Empfänger, Geodaten, Datum und Zeit sowie auch Anhängen.
\footnote{\url{http://www.heise.de/newsticker/meldung/NSA-sammelt-offenbar-fast-200-Millionen-SMS-pro-Tag-2087918.html}}

\subsubsection{Der Fall LavaBit}
LavaBit ist oder besser gesagt war, ein US-amerikanischen Unternehmen dass einen sicheren Webmail-Service zur Verfügung stellte.
Mit ``sicher'' ist gemeint, dass der Dienst es dem Benutzer ermöglichte seine E-Mails asymmetrisch zu verschlüsseln.\footnote{\url{http://de.wikipedia.org/wiki/Asymmetrisches_Kryptosystem}}
Als am 10. Juni 2013 die Identität des NSA-Whisteblowers bekannt wurde, verlangte die US-amerikanische Regierung die Herausgabe von Informationen eines Benutzerkontos von LavaBit.
Sie verlangten alle verfügbaren Adressen, Aufzeichnungen von Sitzungen, Telefonnummern, MAC-Adressen, Bank- und Kreditkartendaten und weiterss von LavaBit Gründer Ladar Levison. Zudem stellte die Regierung auch die Forderung ein Überwachungsgerät installieren zu dürfen, dass diese Daten direkt an sie übermitteln könne. Alle diese Forderungen an Levison wurden von ihm abgelehnt. Er war nicht bereit dazu die Daten der Kunden seiner Firma freizugeben, wodurch er sich vor Gericht verantworten musste.
Ein paar Tage später wurde sogar verlangt, dass er alle öffentlichen sowie auch privaten SSL-Schlüssel der Regierung zu übergeben habe. Nach weiterem hin und her leistete Levsion Folge und druckte die Schlüssel auf Papier mit Schriftgrösse 4 aus. Natürlich ist es kaum möglich diese Daten aus dem Papier zu lesen, so berichtete das FBI, dass die Daten grösstenteils unleserlich waren und nicht gebraucht werden konnten. Das FBI verlangte eine digitale Abgabe aller Schlüssel.
Am 8. August 2013 sah sich Levison als gezwungen seinen Dienst zu schliessen, um die Nutzer nicht an die Strafverfolger ausliefern zu müssen.
Trotz der Schwärzung des Namens vom Inhaber des E-Mail Kontos ist es fast ohne Zweifel, dass es sich um das private E-Mail Konto von Edward Snowden gehandelt hatte.
\footnote{\url{http://www.heise.de/security/meldung/Kritik-an-Lavabits-Konzept-fuer-sichere-E-Mail-2041924.html}}\footnote{\url{http://www.spiegel.de/netzwelt/netzpolitik/dark-mail-alliance-lavabit-und-silent-circle-planen-e-mail-standard-a-931026.html}}\footnote{\url{http://www.heise.de/newsticker/meldung/NSA-Affaere-E-Mail-Anbieter-Lavabit-lieferte-sich-Katz-und-Maus-Spiel-mit-US-Justiz-1972173.html}}\footnote{\url{http://en.wikipedia.org/wiki/Lavabit}}

\subsection{Was meinen die Volksvertreter dazu?}
Die Politiker und Regierungsvertreter von betroffenen Ländern, darunter Deutschland, Frankreich und Brasilien, haben in Bezug auf die unzähligen Enthüllunen, die seit Juni 2013 veröffentlicht wurden, die unterschiedlichsten Reaktionen gezeigt. Wenig überraschend ist es, dass zu Beginn alle schwer empört waren und Stellungnahmen seitens der Verantwortlichen verlangten. Und genauso schnell erklärte man die ganze Affäre für geklärt und versuchte, die aufgebrachte Meute zu beschwichtigen. Das geht aber nicht ganz so einfach, denn die konstante Flut an neuen Enthüllungen lässt sich nicht stoppen. Die amerikanische Regierung konnte offiziell bis Dezember 2013 nicht ermittlen, wie gross das Datenleck ist, das Edward Snowden verursacht hat.
\footnote{\url{http://www.spiegel.de/netzwelt/netzpolitik/nsa-raetselt-ueber-ausmass-der-snowden-enthuellungen-a-939145.html}}
Das führt dazu, dass sich die Politiker, die echte Besorgnis gezeigt haben, in ihrer Meinung unterstützt finden und sich die ``Schönredner'' und Geheimdienstfreunde zwangsläufig mit der Problematik auseinandersetzen müssen.
Hinter Politik stecken immer auch Interessen. Man darf sich deshalb nicht zu sehr auf Aussagen von Schlüsselfiguren in der Weltpolitik verlassen, ganz egal, ob sie sich für oder gegen die Geheimdiesnte aussprechen. Viel mehr sollten die resultierenden Massnahmen als Mass genommen werden.

\subsection{Reaktionen der Bürger}
Die Volksvertreter sind die Vertreter des Volkes. Viel wichtiger als die Meinungen der Politiker, sind demnach die Meinungen des Bürger. Wenn wir hier von Bürgern sprechen, beziehen wir uns nicht auf ein spezifisches Land. Viel mehr meinen wir damit alle in den betroffenen Ländern lebenden Menschen. Wie nicht anders zu erwarten, gibt es auch hier Befürworter und Gegner. Aus Umfragen hat man herauslesen können, dass es viele gibt, die ihre Privatsphäre ohne grosse Bedenken hergeben, solange sie dafür Sicherheit erhalten. Auch fühlen sich viele gar nicht betroffen und glauben, dass ihnen durch die massenhafte Abhörung keine Nachteile entstehen.
\footnote{\url{http://www.faz.net/aktuell/politik/bundestagswahl/neue-allensbach-analyse-wirkungslose-aufregung-12539865.html}}
\\
Man darf nicht vergessen, welche Faktoren bei solchen Überlegungen alle eine Rolle spielen. Viele wollen schlicht nicht auf die unzähligen Dienstleistungen verzichten, die mit grosser Warscheinlichkeit betroffen sind. Die heutige Kombination von sozialen Medien, ständig kommunizierenden Smartphones und Unterhaltung ist schliesslich ein immer festerer Bestandteil unseres Lebens. Nur ungern möchte man einen grösseren Teil des Komforts und die Möglichkeiten hingeben, um mehr Privatsphäre zu haben. Dazu kommt noch, dass man ja nicht mitkriegt, wenn man überwacht wird und oft denkt der Durchschnittsmensch, dass er ja wohl kaum von Interesse ist für die Geheimdienste dieser Welt. Und falls doch, dann hat er ja nichts zu verbergen.
\\
Es gibt aber auch viele Menschen, die sich in Folge der Enthüllungen äusserst besorgt zeigten und auf die Strasse gingen. Es gab Demonstrationen, unzählige Diskussionen in Foren und auf Newsseiten und andere Protestaktionen. Vor allem aus der IT-Welt hagelt es Kritik gegen die Machenschaften der Geheimdienste. Auf welchen Ebenen die Problematik angegangen wird, wird im nachfolgenden Abschnitt näher erläutert.
