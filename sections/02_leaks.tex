\section{Edward Snowden Leaks}
Während den Veröffentlichungen Edward Snowden's mussten eine Menge namhafter Firmen zusehen, wie teils für sie sehr unangenehme Wahrheiten ans Licht gebracht wurden. Wahrheiten, die wohl selbst treue Kunden nachgedenklich gestimmt haben: Soll ich denn wirklich noch meine E-Mails bei Google verwalten? Kann ich noch im Internet surfen, ohne ausspioniert zu werden? Diese Fragen werden zu Recht gestellt, denn es ist wichtig, selbst namhaften Firmen gegenüber kritisch zu sein. Ausserdem kam es in Folge der Veröffentlichung des brisanten Materials zu interessanten Ereignissen, die zugleich besorgniserregend sind. In den folgenden Abschnitten soll als Einstieg in das Thema dieser Arbeit detaillierter auf Teile des veröffentlichten Materials und bestimmte Ereignisse eingegangen werden.

\subsection{Google, Yahoo, Microsoft \& Co. - Kunden werden belauscht}
Noch vor ein paar Monaten dachte wohl noch so mancher Kunde, der ein E-Mail-Konto bei Google, Yahoo oder Microsoft hatte, dass diese nur für ihn selbst einsehbar wären. Man hat ja schliesslich ein sicheres Passwort gewählt, das auch den Vorgaben des entsprechenden Providers entspricht. Nach all den Veröffentlichungen von Edward Snowden wurde aber klar, dass man selbst mit einem guten Passwort keineswegs sicher ist, denn der Provider hat die Kontrolle über die Server, auf denen die Mails gespeichert werden. Was, wenn dieser die Daten seiner Kunden weitergibt? Die Privatsphäre würde verletzt und man hätte praktisch keine Chance, dem zu entgehen, geschweige denn, überhaupt etwas davon zu merken. Doch selbst wenn der Provider dies nicht tut, besteht die Gefahr, dass Geheimdienste oder andere Organisationen beginnen, sich in oben genannte Dienste einzuklinken und private, kundenspezifische Daten abzuschöpfen. Dies war beispielsweise der Fall mit einem Ausspäh-Werkzeug der NSA mit dem Namen \textit{Muscular}. Mit diesem Werkzeug war es der NSA möglich, sich in die Leitungen von Google einzuklinken und Daten mehrerer hundert Millionen Nutzerkonten abzugreifen. Doch nicht nur Google war davon betroffen. Auch Yahoo wurden auf diesem Weg private Nutzerdaten entzogen. Laut den NSA-Papieren vom 9. Januar 2013 hat die NSA innerhalb eines Monats über 181 Millionen Datensätze von Google und Yahoo an Datenzentren des NSA-Hauptquartiers geschickt. Erwähnenswert ist dabei, dass sich die NSA für dieses Vorhaben mit dem britischen Geheimdienst GCHQ zusammengetan hat. Sich also nur auf amerikanische Geheimdienste zu fokusieren, wäre ein Fehler. Natürlich wurde das Programm von NSA-Chef General Keith B. Alexander vehement abgestritten: \textit{``Wir haben keinen Zugang zu Google-Servern, Yahoo-Servern und so weiter.''}\footnote{Link 4, Linkverzeichnis, n-tv}
\\
\\
Doch nicht nur E-Mail-Konten waren betroffen. Laut einem Artikel des britischen Guardian hat die NSA in Zusammenarbeit mit dem GCHQ ein Programm namens \textit{Dishfire} entwickelt, das \textit{so ziemlich alles sammelt, was es findet}. Informationen werden wahllos aus Reiseplänen, Adressbüchern, Finanztransaktionen und weiteren Ressourcen zusammengetragen und ausgewertet. Das Dokument aus dem Jahr 2011, auf das sich der Artikel bezieht, macht deutlich, dass pro Tag zum Teil bis zu 200 Millionen SMS gesammelt wurden. Diese SMS waren verknüpft mit Informationen zu Sender und Empfänger, Geodaten, Datum und Zeit und angehängten Dateien.\footnote{Link 5, Linkverzeichnis, heise}

\subsection{Der Fall LavaBit}
LavaBit ist - oder besser gesagt war - ein US-amerikanisches Unternehmen, welches einen sicheren Webmail-Service zur Verfügung gestellt hatte. Mit \textit{sicher} ist gemeint, dass der Dienst dem Kunden die Möglichkeit bot, seine E-Mails zu verschlüsseln. Als am 10. Juni 2013 die Identität des NSA-Whisteblowers Edward Snowden bekannt wurde, verlangte die US-amerikanische Regierung die Herausgabe von Informationen eines bestimmten Benutzerkontos von LavaBit. Zu den verlangten Informationen gehörten Adressen, Aufzeichnungen von Sitzungen, Telefonnummern, MAC-Adressen sowie Bank- und Kreditkartendaten. Zudem stellte die Regierung die Forderung, ein Überwachungsgerät installieren zu dürfen, welches diese Daten direkt an sie übermitteln sollte. LavaBit-Gründer Ladar Levison lehnte jedoch alle Forderungen ab. Er war nicht bereit, die Daten seines Kunden freizugeben. Kurz darauf wurde Levison aufgefordert, alle öffentlichen und privaten SSL-Schlüssel seines Dienstes der Regierung zu übergeben. Dies hätte mit ziemlicher Wahrscheinlichkeit eine komplette \textit{De-Anonymisierung} aller seiner Kunden bedeutet. Nach weiterem Hin und Her leistete Levison schliesslich Folge und druckte die Schlüssel auf Papier mit der Schriftgrösse vier aus. Das FBI berichtete, dass die Daten auf Grund der sehr kleinen Schrifftgrösse mehrheitlich unleserlich waren und nicht gebraucht werden konnten. Sie verlangten darauf eine Abgabe der Schlüssel in digitaler Form. Um die Nutzer seines Webmail-Services nicht an die Strafverfolger ausliefern zu müssen, sah sich Levison gezwungen, seinen Dienst per 8. August 2013 zu schliessen. Trotz der Schwärzung des Namens vom Inhaber des ursprünglich betroffenen E-Mail-Kontos ist es fast ohne Zweifel, dass es sich um das private E-Mail-Konto von Edward Snowden gehandelt hatte. Levison selbst bekam während der Geschehnisse auf Grund seines Widerstands Probleme mit der Justiz und musste sich unter anderem vor Gericht verantworten.
\footnote{Link 6, Linkverzeichnis, heise}
\footnote{Link 7, Linkverzeichnis, Spiegel}
\footnote{Link 8, Linkverzeichnis, heise}
\footnote{Link 9, Linkverzeichnis, Wikipedia}

\subsection{Weite Anlaufstellen}
Die Abschnitte in diesem Kapitel haben nur einen sehr kleinen Teil der Enthüllungen und der damit zusammenhängenden Ereignisse abgedeckt. Da der Fokus dieser Arbeit ein anderer ist, wird an dieser Stelle auch nicht detaillierter darauf eingegangen. Es ist aber wärmstens empfohlen, bei Interesse das Internet nach weiteren Informationen zu durchforsten, denn alle Enthüllungen sind von grosser Wichtigkeit und hochinteressant dazu. Vor allem auf IT-Seiten wird man schnell fündig. Als Beispiel soll an dieser Stelle der bekannte Nachrichtenticker \textit{heise online} genannt werden. Dieser hat die NSA-Affäre zu einem Topthema erklärt und ist seit den ersten Enthüllungen bis heute am Ball geblieben. Zum Einstieg bietet er eine Übersichtsseite zu den Enthüllungen an. Man findet diese unter: \url{http://www.heise.de/thema/NSA}. Es wurde darüber hinaus eine \textit{Time-Line} eingerichtet, der man von den ersten Enthüllungen bis zu den brandaktuellen Neuigkeiten folgen kann: \url{http://www.heise.de/extras/timeline}
