\newpage
\section{Edward Snowden Leaks}
Die Masse der im Laufe der Snowden-Enthüllungen veröffentlichten Zeitungsartikel, Newsartikel, Grafiken, Videos und sonstigen Beiträge im Netz ist enorm.
Vor allem IT-Seiten wie beispielsweise ``heise online'', ein Nachrichtenticker mit Schwerpunkt Informations- und Telekommunikationstechnik, haben sich in den letzten Monaten intensiv dem Thema gewidmet.

\subsection{Die Enthüllungen}
Eine Menge namhafter Firmen mussten zusehen wie teils, für sie, sehr unangenehme Wahrheiten ans Licht gebracht wurden. Wahrheiten, die wohl selbst treue Kunden nachgedenklich gestimmt haben: Soll ich denn wirklich noch meine E-Mails bei Google verwalten? Kann ich noch im Internet surfen, ohne ausspioniert zu werden?. Diese Fragen werden zu Recht gestellt, denn es ist wichtig, selbst namhaften Firmen gegenüber kritisch zu sein.

\subsubsection{Google, Yahoo, Microsoft \& Co. - Kunden werden belauscht}
Noch vor ein paar Monaten dachte wohl noch so mancher, der ein E-Mail-Konto bei Google, Yahoo oder Microsoft hatte, dass diese nur für ihn selbst einsichtbar wären. Man hat ja schliesslich ein sicheres Passwort gewählt, das auch den Vorgaben des entsprechenden Providers entspricht. Nach all den Veröffentlichungen von Edward Snowden wurde aber klar, dass man selbst mit einem guten Passwort keineswegs sicher ist. Der Provider hat die Kontrolle über die Server, auf denen die Mails gespeichert werden. Was, wenn der Provider diese Daten weitergibt? Die Privatspähre wird missbraucht und man hat praktisch keine Chance, dem zu entgehen, geschweige denn, überhaupt etwas davon mitzukriegen.
Doch selbst wenn der Provider dies nicht tut, besteht dennoch die Gefahr, dass Geheimdienste oder andere Organisationen beginnen, sich in oben genannte Dienste einzuklinken oder auf sonstige Art an die Daten zu gelangen.
Dies war beispielsweise der Fall mit dem Ausspäh-Werkzeug ``Muscular'' der NSA. Mit diesem Werkzeug war es der NSA möglich sich in die Leitungen von Google einzuklinken und Daten mehrerer hundert Millionen Nutzerkonten abzugreifen. Nicht nur Google war davon betroffen. Auch Yahoo wurden auf diesem Weg private Nutzerdaten entzogen. Laut den NSA-Papieren vom 9. Januar 2013 hat die NSA innert eines Monates über 181 Millionen Datensätze von Google und Yahoo an Datenzentren des NSA-Hauptquartiers geschickt. Dazu muss noch gesagt werden, dass sich die NSA für dieses Werkzeug mit dem britischen Geheimdienst ``GCHQ'' zusammen getan hat.
Natürlich wurde dies von NSA-Chef General Keith B. Alexander vehement abgestritten: ``Wir haben keinen Zugang zu Google-Servern, Yahoo-Servern und so weiter.''\footnote{\url{http://www.n-tv.de/politik/Google-kritisiert-Ausspaehung-durch-NSA-Gmail-Docs-und-Maps-offenbar-betroffen-article11640736.html}} 
\\
\\
Doch nicht nur E-Mail Konten waren betroffen. Laut einem Artikel des britischen Guardian hat die NSA in Zusammenarbeit mit dem GCHQ ein Programm namens ``Dishfire'' entwickelt, das ``so ziemlich alles sammelt, was es findet''.
Informationen werden wahllos aus Reiseplänen, Adressbüchern, Finanztransaktionen und weiteren Ressourcen zusammengetragen und ausgewertet. Das Dokument vom Jahr 2011, auf das sich der Artikel bezieht, macht deutlich, dass pro Tag zum Teil bis zu 200 Millionen SMS gesammelt wurden. Diese SMS waren Verknüpft mit dem Sender und Empfänger, Geodaten, Datum und Zeit sowie auch Anhängen.\footnote{\url{http://www.heise.de/newsticker/meldung/NSA-sammelt-offenbar-fast-200-Millionen-SMS-pro-Tag-2087918.html}}

\subsubsection{Der Fall LavaBit}
LavaBit ist oder, besser gesagt, war ein US-amerikanischen Unternehmen, welches einen sicheren Webmail-Service zur Verfügung stellte.
Mit ``sicher'' ist gemeint, dass der Dienst dem Benutzer die Möglichkeit bot, seine E-Mails asymmetrisch zu verschlüsseln.\footnote{\url{http://de.wikipedia.org/wiki/Asymmetrisches_Kryptosystem}}
Als am 10. Juni 2013 die Identität des NSA-Whisteblowers bekannt wurde, verlangte die US-amerikanische Regierung die Herausgabe von Informationen eines Benutzerkontos von LavaBit.
Sie verlangten alle verfügbaren Adressen, Aufzeichnungen von Sitzungen, Telefonnummern, MAC-Adressen, Bank- und Kreditkartendaten und weitere Daten von LavaBit-Gründer Ladar Levison. Zudem stellte die Regierung auch die Forderung, ein Überwachungsgerät installieren zu dürfen, das diese Daten direkt an sie übermitteln könne. Alle diese Forderungen an Levison wurden von ihm abgelehnt. Er war nicht bereit, die Daten seiner Kunden freizugeben, wodurch er sich vor Gericht verantworten musste.
Ein paar Tage später wurde sogar verlangt, dass er alle öffentlichen und privaten SSL-Schlüssel der Regierung zu übergeben habe. Nach weiterem Hin und Her leistete Levsion Folge und druckte die Schlüssel auf Papier in Schriftgrösse 4 aus. Natürlich ist es kaum möglich, diese Daten auf dem Papier zu lesen. Das FBI berichtete, dass die Daten grösstenteils unleserlich waren und nicht gebraucht werden konnten. Sie verlangten darauf eine Abgabe der Schlüssel in digitaler Form.
Am 8. August 2013 sah sich Levison gezwungen, seinen Dienst zu schliessen, um die Nutzer nicht an die Strafverfolger ausliefern zu müssen.
Trotz der Schwärzung des Namens vom Inhaber des E-Mail Kontos, ist es fast ohne Zweifel, dass es sich um das private E-Mail Konto von Edward Snowden gehandelt hatte.\footnote{\url{http://www.heise.de/security/meldung/Kritik-an-Lavabits-Konzept-fuer-sichere-E-Mail-2041924.html}}
\footnote{\url{http://www.spiegel.de/netzwelt/netzpolitik/dark-mail-alliance-lavabit-und-silent-circle-planen-e-mail-standard-a-931026.html}}
\footnote{\url{http://www.heise.de/newsticker/meldung/NSA-Affaere-E-Mail-Anbieter-Lavabit-lieferte-sich-Katz-und-Maus-Spiel-mit-US-Justiz-1972173.html}}
\footnote{\url{http://en.wikipedia.org/wiki/Lavabit}}

\subsection{Was meinen die Volksvertreter dazu?}
Die Politiker und Regierungsvertreter von betroffenen Ländern, darunter Deutschland, Frankreich und Brasilien, haben in Bezug auf die unzähligen Enthüllunen, die seit Juni 2013 veröffentlicht wurden, die unterschiedlichsten Reaktionen gezeigt. Es war vorauszusehen, dass zu Beginn alle schwer empört waren und Stellungnahmen seitens der Verantwortlichen verlangten. Und genauso schnell erklärte man die ganze Affäre für geklärt und versuchte, die aufgebrachten Bürger und Politiker zu beschwichtigen. Das geht aber nicht ganz so einfach, denn die konstante Flut an neuen Enthüllungen lässt sich nicht stoppen. Die amerikanische Regierung konnte offiziell bis Dezember 2013 nicht ermittlen, wie gross das Datenleck ist, das Edward Snowden hinterlassen hat.\footnote{\url{http://www.spiegel.de/netzwelt/netzpolitik/nsa-raetselt-ueber-ausmass-der-snowden-enthuellungen-a-939145.html}}
Das führt dazu, dass sich die Politiker, die echte Besorgnis gezeigt haben, in ihrer Meinung unterstützt finden und sich die ``Schönredner'' und Geheimdienstfreunde zwangsläufig mit der Problematik auseinandersetzen müssen.
Hinter Politik stecken bekanntlich immer auch Interessen. Man darf sich deshalb nicht zu sehr auf Aussagen von Schlüsselfiguren in der Weltpolitik verlassen, ganz egal, ob sie sich für oder gegen die Geheimdiesnte aussprechen. Viel mehr sollten die resultierenden Massnahmen als Mass genommen werden.

\subsection{Reaktionen der Bürger}
Die Volksvertreter sind, wie der Name schon sagt, die Vertreter des Volkes. Viel wichtiger als die Meinungen der Politiker sind demnach die Meinungen der Bürger. Mit Bürgern sind an dieser Stelle alle in den betroffenen Ländern lebenden Menschen gemeint. Wie nicht anders zu erwarten, gibt es auch hier Befürworter und Gegner. Aus Umfragen hat man herauslesen können, dass es viele gibt, die ihre Privatsphäre ohne grosse Bedenken hergeben, solange sie dafür Sicherheit erhalten. Auch fühlen sich viele gar nicht betroffen und glauben, dass ihnen durch die massenhafte Abhörung kein (unmittelbarer) Schaden entsteht.\footnote{\url{http://www.faz.net/aktuell/politik/bundestagswahl/neue-allensbach-analyse-wirkungslose-aufregung-12539865.html}}
\\
Man darf nicht vergessen, welche Faktoren bei solchen Überlegungen eine Rolle spielen. Viele wollen schlicht nicht auf die unzähligen Dienstleistungen verzichten, die mit grosser Warscheinlichkeit betroffen sind. Die Kombination von sozialen Medien, ständig kommunizierenden Smartphones und Unterhaltung ist schliesslich ein immer festerer Bestandteil des heutigen Lebens. Nur ungern möchte man einen Teil des Komforts hingeben, um mehr Privatsphäre zu haben. Dazu kommt noch, dass man ja nicht mitkriegt, wenn man überwacht wird. Der Durchschnittsmensch denkt oft, dass er ja wohl kaum von Interesse ist für die Geheimdienste dieser Welt. Und falls doch, dann hat er ja nichts zu verbergen.
\\
Es gibt aber auch viele Menschen, die sich in Folge der Enthüllungen äusserst besorgt zeigten und ihren Sorgen Gehör verschafften. Es gab Demonstrationen, unzählige Diskussionen in Foren und auf Newsseiten und andere Protestaktionen. Vor allem aus der IT-Welt hagelt es Kritik gegen die Machenschaften der Geheimdienste. Auf welchen Ebenen die Problematik angegangen wird, wird im nachfolgenden Abschnitt näher erläutert.

\subsection{Reaktionen aus der IT-Welt}
... to be implemented

\subsection{Weite Anlaufstellen}
Die Abschnitte in diesem Kapitel haben nur einen kleinen Teil dessen abgedeckt, was es zu wissen gibt. Bei Interesse lohnt es sich, im Internet nach weiteren Informationen zu suchen. Vor allem auf IT-Seiten wird man schnell fündig. Der bereits genannte Nachrichtenticker ``heise online'' hat die NSA-Affäre zu einem Topthema erklärt und bietet eine Übersichtsseite zu den Enthüllungen unter: \url{http://www.heise.de/thema/NSA}. Es wurde sogar eine Time-Line eingerichtet, der man von den ersten Enthüllungen bis zu den brandaktuellen Neuigkeiten folgen kann: \url{http://www.heise.de/extras/timeline/#vars!date=2013-06-06_10:12:00!}