\begin{itemize}
  \item Raspberry Pi, Model B
  \item Micro-USB Kabel für Stromversorgung
  \item SD Speicherkarte
  \item RJ45 Netzwerkkabel
  \item WiFi Adapter
\end{itemize}

Es gibt gewisse Voraussetzungen, die man bei der Auswahl der Komponenten beachten muss. Das Micro-USB Kabel für die Stromversorgung muss folgende Spezifikationen erfüllen: mind. 700mA, 5V. Die SD-Karte, auf die später das Betriebssystem geladen wird, sollte mindestens eine Kapazität von 4 Gigabyte aufweisen. Zudem muss man beachten, dass nicht jede verfügbare Karte auch tatsächlich unter dem verwendeten Betriebssystem funktioniert. Deshalb gibt es im Internet eine Liste, auf der eine Reihe kompatibler und nicht kompatibler Karten aufgeführt werden: \url{http://elinux.org/RPi\_SD\_cards\#Working\_.2F\_Non-working\_SD\_cards}.
Das Netzwerkkabel wird verwendet, um den Raspberry Pi mit dem Internet zu verbinden. Dies kann beispielsweise über einen Switch, Router oder RJ45-Dose geschehen.
Bei der Auswahl des WiFi Adapters ist auch wieder Vorsicht geboten, denn nicht alle im Handel erhältlichen Adapter sind mit dem Raspberry Pi kompatibel. Eine Liste, welche Adapter benutzt werden können gibt es hier: \url{http://elinux.org/RPi_USB_Wi-Fi_Adapters}.
\\
Für die Einrichtung selbst wird noch zusätzliche Peripherie gebraucht, die aber in den meisten Haushalten schon vorhanden sein sollte:

\begin{itemize}
  \item Computer oder Notebook
  \item SD-Karten-Leser
  \item Monitor mit HDMI-Anschluss
  \item HDMI-Kabel
  \item Tastatur
  \item Maus
\end{itemize}

Die aufgelisteten Peripherie braucht man nur für die Einrichtung selbst oder anfallende Wartungsarbeiten.
Ist alles erst einmal korrekt eingerichtet, kann diese wieder andersweitig verwendet werden.
\\
Detaillierte Informationen und Tipps zu den Komponenten sowie der Peripherie findet man unter: \url{http://www.raspberrypi.org/phpBB3/viewtopic.php?t=4277}

\subsection{Betriebssystem installieren}
Bevor der Computer gestartet und mit der Installation begonnen werden kann, muss noch das Betriebssystem installiert werden.
Die SD Speicherkarte kann mit wenigen Anweisungen von jedem gängigen Betriebssystem (Windows, OSX, Linux) aus mit einem Raspbian bestückt werden.
\\
\\
Zuerst gilt es das Betriebssystem selbst aus dem Internet zu laden.
\footnote{Das Betriebssystem kann unter \url{http://www.raspberrypi.org/downloads} heruntergeladen werden}.
Ist die Datei vollständig heruntergeladen und die SD-Karte angeschlossen, kann das heruntergeladene Image (Datei) auf die Karte geladen werden.
Es gibt eine sehr gute Anleitung im Internet, die die einzelnen Schritte in Details beschreibt. Man findet sie unter: \url{http://elinux.org/RPi\_Easy\_SD\_Card\_Setup}.
