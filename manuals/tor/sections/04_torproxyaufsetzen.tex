\section{WiFi Adapter teste}
Es lohnt sich, zuerst überprüfen, ob der angeschaffte WiFi-Adapter auf dem aufgespielten Betriebssystem lauffähig ist. Dazu muss man lediglich den Adapter in den USB-Port stecken und nach kurzer Wartezeit folgendes Kommando aufrufen:

\begin{lstlisting}
ifconfig -a
\end{lstlisting}

Wenn auf der erschienen Ausgabe einen Eintrag für \textit{wlan0} zu sehen ist, kann mit der Einrichtung des Access Points begonnen werden.

% FIXME: Bild von ifconfig?

\section{Access Point einrichten}
Mit dem eingesteckten Netzwerkkabel und dem per USB angeschlossenen WiFi-Adapter stehen nun zwei Schnittstellen zur Verfügung, um sich mit dem Internet zu verbinden. Das Netzwerkkabel hat die Aufgabe, den Raspberry Pi mit dem Internet zu verbinden. Der WiFi-Adapter hingegen soll so eingerichtet werden, dass er ein lokales WLAN aufzieht.
Hierzu wird ein eigener DHCP-Server benögtigt, welcher dafür verantwortlich ist, dass jeder Computer, der sich mit dem WLAN verbindet, eine valide IP-Adresse erhält und somit ein Teil des Netzes werden kann.

% FIXME: Acces Point erklären??

Dieser und die Software, welche als \textit{Access Point} selbst agiert, müssen durch folgenden Befehl installiert werden:
 
\begin{lstlisting}
apt-get install hostapd isc-dhcp-server
\end{lstlisting}

\subsection{DHCP-Server konfigurieren}
Damit der DHCP-Server auch richtig funktioniert, muss er zuerst ein wenig konfiguriert werden.
Dazu öffnen man die Datei \textit{/etc/dhcp/dhcpd.conf} mit einem Texteditor. Hier wird dazu der vorinstallierte Texteditor Nano verwendet.

\begin{lstlisting}
nano /etc/dhcp/dhcpd.conf
\end{lstlisting}

Zuerst müssen die folgenden Zeile gefunden und mit einer vorangestellten Raute (\#) auskommentiert werden. Auskommentiert bedeutet, dass die Zeile ungültig und somit nicht mehr aktiv ist:

\begin{lstlisting}
option domain-name "example.org"
option domain-name-server ns1.example.org,ns2.example.org
\end{lstlisting}

Neu sieht das ganze folgendermassen aus:

\begin{lstlisting}
#option domain-name "example.org"
#option domain-name-server ns1.example.org,ns2.example.org
\end{lstlisting}

Als nächstes muss dem installierten DHCP-Server gesagt werden, dass er der offizielle DHCP-Server dieses Netzes ist. Einmal eingestellt vergibt er fortan valide IP-Adressen an jeden Client, der ihn mittels DHCP-Request anfragt.
Dazu muss folgende Zeile einkommentiert - die vorangestellte Raute entfernt - werden. Aus

\begin{lstlisting}
#authoritative;
\end{lstlisting}

wird somit:

\begin{lstlisting}
authoritative;
\end{lstlisting}

Jetzt gilt es, alle Einstellungen des vom WiFi-Adapter künftig aufgezogenen WLANs festzulegen. 
An Ende der Datei müssen dazu folgende Zeilen angefügt werden:

\begin{lstlisting}
subnet 192.168.66.0 netmask 255.255.255.0 {
  range 192.168.42.10 192.168.42.50;
  option broadcast-address 192.168.66.255;
  option routers 192.168.66.1;
  default-lease-time 666;
  max-lease-time 7200;
  option domain-name "local";
  option domain-name-servers 8.8.8.8, 8.8.4.4;
}
\end{lstlisting}

Den DHCP-Server ist nun fertig konfiguriert. Man muss ihn noch an ein Netwerkinterface binden. Dieses Netzwerkinterface ist in diesem Fall wlan0, wie schon zu beginn mit ``ifconfig'' herausgefunden werden konnte.
In der Datei \textit{/etc/default/isc-dhcp-server} muss dazu bei der Einstellung \textit{INTERFACES} wlan0 eingetragen werden:

\begin{lstlisting}
INTERFACES="wlan0"
\end{lstlisting}

Dem Interface \textit{wlan0} kann jetzt eine statische IP-Adresse vergeben wird. Es handelt sich dabei um jene Adresse, die bei der Konfiguration des DHCP-Server bei \textit{option routers} angegeben wurde. \\
In der Datei \textit{/etc/network/interfaces} müssen die Zeilen:

\begin{lstlisting}
iface wlan0 inet manual
wpa-roam: /etc/etc/wpa_supplicant/wpa_supplicant.conf
iface default inet dhcp
\end{lstlisting}

Mit den folgenden ersetzt werden:

\begin{lstlisting}
iface wlan0 inet static
  address 192.168.66.1
  netmask 255.255.255.0
\end{lstlisting}

Das Interface \textit{wlan0} läuft zu diesem Zeitpunkt bereits. Damit die Änderung sofort wirksam wird, kann man dem Interface mit folgendem Befehl die zuvor definierte IP-Adresse aufzwingen:

\begin{lstlisting}
ifconfig wlan0 192.168.66.1
\end{lstlisting}

% FIXME: ifconfig oder ifup?

\subsection{Hostapd konfigurieren}
Weiter geht es darum den Access Point selbst einzurichten. Das verwendete Tool hostapd haben wir bereits installiert.
Die Datei \textit{/etc/hostapd/hostapd.conf} muss erstellt und folgendes eingetragen werden:

\begin{lstlisting}
interface=wlan0
driver=r8712u
ssid=PiProxy
hw_mode=g
channel=6
macaddr_acl=0
auth_algs=1
ignore_broadcast_ssid=0
wpa=2
wpa_passphrase=pi
wpa_key_mgmt=WPA-PSK
wpa_pairwise=TKIP
rsn_pairwise=CCMP
\end{lstlisting}

Der Treiber ist je nach verwendetem WiFi Adapter anders. Was bedeutet, dass dieser zuerst ausfindig gemacht werden muss. \\

Diese nun erstellte Konfiguration muss \textit{hostapd} bekannt gemacht werden. In der Datei \textit{/etc/default/hostapd.conf} muss nach \textit{DAEMON\_CONF} gesucht und diese Zeile wie folgt geändert werden:

\begin{lstlisting}
DAEMON_CONF="/etc/hostapd/hostapd.conf"
\end{lstlisting}

Der Raspberry Pi ist nun soweit ein WiFi Netzwerk erstellen und sich mit dem Internet verbinden zu können. Was noch fehlt ist die Verbindung zwischen dem WiFi-Netz und dem Internet.

\subsection{NAT konfigurieren}
NAT wird gebraucht um die Clients vom WiFi-Netz auf das Internet zu leiten.

Zuerst wird der Datei \textit{/etc/sysctl.conf} folgendes am Ende angefügt:

\begin{lstlisting}
net.ipv4.ip_forward=1
\end{lstlisting}

Und danach soll noch folgendes ausgeführt werden, um das Weiterleiten im laufenden Betrieb vom System zu aktivieren:

\begin{lstlisting}
echo 1 > /proc/sys/net/ipv4/ip_forward
\end{lstlisting}

Weiter muss die Firewall entsprechend angepasst werden, damit NAT das eingerichtet Forwarding ausführen kann.
Dazu folgende Befehle ausführen:

\begin{lstlisting}
iptables -t nat -A POSTROUTING -o eth0 -j MASQUERADE
iptables -A FORWARD -i eth0 -o wlan0 -m state --state RELATED,ESTABLISHED -j ACCEPT
iptables -A FORWARD -i wlan0 -o eth0 -j ACCEPT
\end{lstlisting}

Damit man diese Befehle nicht bei jedem Neustart des Raspberry Pi's eingeben muss, speichern wir diese mit folgendem Befehl permanent:

\begin{lstlisting}
iptables-save > /etc/iptables.ipv4.nat
echo "up iptables-restore < /etc/iptables.ipv4.net" >> /etc/network/interfaces
\end{lstlisting}

\section{Aufstarten von hostapd und unserem DHCP}
Bevor das Wlan-Netz getestet werden kann, müssen alle beteiligten Komponenten aufgestartet werden.
Hier genügen folgende Befehle:

\begin{lstlisting}
service hostapd start
service isc-dhcp-server start
\end{lstlisting}

\section{Tor aufsetzen}
TOR kann man wie die zuvor verwendeten Programme auch über die Paketverwaltung installieren:

\begin{lstlisting}
apt-get install tor
\end{lstlisting}

Nach der Installation muss TOR natürlich noch entsprechend, für unseren Nutzen, konfiguriert werden. \\
Direkt nach dem einleitenden Kommentarblock kommen folgende Konfigurationen:

\begin{lstlisting}
Log notice file /var/log/tor/notices.log
VirtualAddrNetwork 10.192.0.0/10
AutomapHostsSuffixes .onion,.exit
AutomapHostsOnResolve 1
TransPort 9040
TransListenAddress 192.168.66.1
DNSPort 53
DNSListenAddress 192.168.66.1
\end{lstlisting}

Es sind weitere Einstellungen vorzunehmen. Und zwar muss der Firewall beigebracht werden, dass die Kommunikation, welche über das Wlan-Netz kommt, über das TOR-Netzwerk weitergeleitet werden soll. \\
Dazu einfach folgende Befehle ausführen:

%% FIXME: ssh chapter!

\begin{lstlisting}
iptables -F
iptables -t nat -F
iptables -t nat -A PREROUTING -i wlan0 -p tcp --dport 22 -j REDIRECT --to-ports 22
iptables -t nat -A PREROUTING -i wlan0 -p udp --dport 53 -j REDIRECT --to-ports 53
iptables -t nat -A PREROUTING -i wlan0 -p tcp --syn -j REDIRECT --to-ports 9040
iptables-save > /etc/iptables.ipv4.nat
\end{lstlisting}

In der \textit{torrc} Konfigurationsdatei wurde eine Logdatei angegeben, dort werden alle Logs hingeschrieben. \\
Diese Datei muss nun erstellt und mit den entsprechenden Rechten versehrt werden:

\begin{lstlisting}
touch /var/log/tor/notices.log
chown debian-tor /var/log/tor/notices.log
chmod 644 /var/log/tor/notices.log
\end{lstlisting}

Nun kann man TOR mit den neuen Konfigurationen neustarten:

\begin{lstlisting}
service tor restart
\end{lstlisting}

Und um zu prüfen ob TOR auch mit den neuen Konfigurationen erfolgreich gestartet werden konnte, kann man folgendes ausführen:

\begin{lstlisting}
service tor status
\end{lstlisting}

Die Ausgabe sollte ähnlich dieser sein:

% FIXME: screenshot

\section{IP-Adresse verifizieren}
Mittels \url{ipchicken.com} kann die IP-Adresse ermittelt werden, mit der man sich im Internet bewegt. Diese wird einem normalerweise vom Provider zugeteilt und ändert sich nicht so schnell. Mit einem funktionierenden TOR-Setup, ändert sich das Verhalten. Dadurch, das die gesendete Abfrage drei verschiedene TOR-Server passiert, ändert sich die Urpsrungs-IP-Adresse des Paketes jedes Mal. Um dieses Verhalten zu verifizieren, muss man sich lediglich von einem Computer im normalen Netzt und einem Computer vom eigens geschaffenen Wlan-Netz mit IP-Chicken verbinden. Sind die IP-Adressen nicht identisch und steht beim Computer vom Wlan-Netzt nichts vom Provider, läuft TOR erfolgreich.
