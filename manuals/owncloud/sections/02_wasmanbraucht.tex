\section{Was man braucht}
Um OwnCloud auf dem Raspberry Pi einzurichten, benötigt man folgendes:

\begin{itemize}
  \item Raspberry Pi Model B
  \item Micro-USB Kabel für Stromversorgung
  \item SD Speicherkarte
  \item RJ45 Netzwerkkabel
\end{itemize}

Es gibt einige Voraussetzungen, die man zu beachten hat, so wird mindestens ein Micro-USB Kabel dass über 5 Volt versorgt werden kann benötigt. Die SD Karte, auf die später das Betriebssystem geladen wird, sollte mindestens eine Kapazität von 2 Gigabyte Speicherplatz aufweisen. Zudem muss man beachten, dass nicht jede verfügbare Karte auch tatsächlich unter unserem verwendeten Betriebssystem funktioniert. Deshalb gibt es im Internet eine Liste, auf der eine Reihe kompaitbler und nicht kompatibler Karten mit Hinweisen aufgeführt werden: \url{http://elinux.org/RPi\_SD\_cards#Working\_.2F\_Non-working\_SD\_cards}. \\
Das Netzwerkkabel wird verwendet um den Raspberry Pi mit dem Internet zu verbinden. Dies kann beispielsweise über einen Switch, Router oder RJ45-Dose geschehen. \\

\subsection{Das Raspberry Pi Betriebssystem}
Das Betriebssystem, welches üblicherweise für einen Raspberry Pi gebraucht wird, heisst Raspbian. Es basiert auf einem gewohnlichen GNU/Linux Debian System, welches so modifiziert wurde, dass es auf der gegebener Hardware ohne Probleme laufen kann.
Die SD Speicherkarte kann mit wenigen Anweisungen von jedem gängigen Betriebssystem aus mit einem Raspbian bestückt werden.
Zuerst gilt es das Betriebssystem selbst aus dem Internet zu laden\footnote{Das Betriebssystem kann von \url{http://www.raspberrypi.org/downloads} heruntergeladen werden}. Sobald man die SD-Karte an den Computer angeschlossen hat, kann man damit beginnen, das heruntergeladenen Betriebssystem auf die Karte zu speichern.
Es gibt eine sehr gute Anleitung, die die einzelnen Schritte in Details beschreibt unter: \url{http://elinux.org/RPi\_Easy\_SD\_Card\_Setup}
