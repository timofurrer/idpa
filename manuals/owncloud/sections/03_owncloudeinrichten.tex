\section{ownCloud einrichten}
Nachdem das Betriebssystem erfolgreich auf dem Raspberry Pi aufgesetzt werden konnte und alle Kabel und Komponenten angeschlossen sind, kann mit der eigentlichen Installation von ownCloud begonnen werden.

\subsection{System aktualisieren}
Um sicher zu stellen, dass das System auf dem aktuellsten Stand ist, müssen zuerst alle installierten Pakete aktualisiert werden.
Dazu und für alle weiteren Schritte wird ein Terminal benötigt. Dieses öffnet man, indem man das Startmenü am linken unteren Rand öffnet und auf ``Accessories -> LXTerminal'' klickt. Es öffnet sich ein Fenster mit schwarzem Hintergrund, das Terminal.
\\
Folgende Befehle müssen im Terminal abgesetzt werden, um das System zu aktualisieren:
\begin{lstlisting}
sudo apt-get update
sudo apt-get upgrade
\end{lstlisting}

Um ein Befehl abzusetzen, reicht ein Drücken der Enter-Taste am Ende der Zeile.

\subsection{Abhänigkeiten installieren}
Damit ownCloud reibungslos laufen kann, müssen einige zusätzliche Pakete installiert werden.
\\
Die folgenden Befehle können einfach kopiert und ins Terminal eingefügt werden:

\begin{lstlisting}
sudo apt-get install apache2 php5 php5-gd php5-sqlite php5-curl php5-json php5-common php5-intl php-pear php-apc php-xml-parser libapache2-mod-php5 curl libcurl3 libcurl3-dev sqlite
\end{lstlisting}

\subsection{Statische IP-Adresse}
Die IP-Adresse, durch welche der Computer im Netz eindeutig identifiziert werden kann und über welche letztendlich der Zugriff auf die Cloud erfolgen wird, kann nach jedem Computerstart anderst aussehen.
Aus diesem Grund ist es sinnvoll, dass die IP-Adresse statisch festgelegt wird und somit immer gleich bleibt.
Auf unserem Raspbian kann dies mit folgenden Schritten bewerkstelligt werden:

\begin{itemize}
  \item Die Netwerkkonfigurationsdatei unter ``/etc/network/interfaces'' muss angepasst werden.\\ Die Datei muss mit einem Texteditor geöffnet:
    \begin{lstlisting}
sudo nano /etc/network/interfaces
    \end{lstlisting} 
Und folgende Zeilen angefügt werden:
    \begin{lstlisting}
iface eth0 inet static
address 192.168.1.99
netmask 255.255.255.0
gateway 192.168.1.1
network 192.168.1.0
broadcast 192.168.1.255
    \end{lstlisting}
Die verwendete IP-Adresse in der zweiten Zeile (192.168.1.99) sollte durch die aktuelle IP-Adresse des Systems ersetzt werden, sofern nichts anderes gewünscht wird. Diese kann mit dem Kommando ``ifconfig'', welches in ein Terminal eingegeben wird, ermittelt werden:
    % FIXME: placholder for ifconfig screenshot
  \item Nun muss das Interface noch neu gestartet werden, damit die Änderungen wirksam werden:
    \begin{lstlisting}
sudo ifdown eth0
sudo ifup eth0
    \end{lstlisting}
\end{itemize}

\subsection{Webserver-Benutzer anlegen}
Aus Sicherheitsgründen sollte ein Webserver immer von einem Benutzer mit eingeschränkten Rechten ausgeführt werden. Er soll nur auf jene Dateien zugreifen können, die er auch wirklich braucht. Dazu legt man einen Benutzer namens ``www-data' an und fügt ihn anschliessend der gleichnamigen Gruppe zu:

\begin{lstlisting}
sudo groupadd www-data
sudo usermod -a -G www-data www-data
\end{lstlisting}

\subsection{Webserver konfigurieren}
Als Webserver wird apache2 verwendet. Dieser wurde bereits im Abschnitt ``Abhängigkeiten installieren'' installiert. Er muss jetzt lediglich noch so eingerichtet werden, 
dass ownCloud darauf laufen kann.

\begin{itemize}
  \item Apache verlangt beim Start zu wissen, wie der Webserver heisst. Man hat hier freie Wahl, sollte aber etwas Sinnvolles eingeben, wie beispielsweise ``owncloud''.
  \subitem Jetzt muss der Name noch der Konfigurationsdatei von Apache hinzgefügt werden:
    \begin{lstlisting}
sudo echo "ServerName owncloud" >> /etc/apache2/apache2.conf
    \end{lstlisting}
  \item Auch dem System selbst muss dieser Namen bekannt gegeben werden. Folgender Befehl fügt die entsprechende Zeile der Datei ``/etc/hosts'' hinzu:
    \begin{lstlisting}
sudo echo "127.0.0.1 ownloud" >> /etc/hosts
    \end{lstlisting}
  \item Damit owncloud korrekt ausgeführt werden darf, müssen ihm noch gewisse Rechte eingeräumt werden.
    Dazu öffnet man die Apache Konfigurationsdatei ``/etc/apache2/sites-enabled/000-default'' mit einem Texteditor:
    \begin{lstlisting}
sudo nano /etc/apache2/sites-enabled/000-default
    \end{lstlisting}
    Und sucht nach folgendem Abschnitt:
    \begin{lstlisting}
<Directory /var/www/>
  Options Indexes FollowSymLinks MultiViews
  AllowOverride All
  Order allow,deny
  allow from all
</Directory>
    \end{lstlisting}
    Dort muss man die Zeile ``AllowOverride None'' mit ``AllowOverride All'' ersetzen.
  \item Zum Schluss müssen noch ein paar zusätzliche Module für den Webserver installiert werden:
    \begin{lstlisting}
sudo a2enmod rewrite
sudo a2enmod headers
    \end{lstlisting}
\end{itemize}

\subsection{Webserver absichern}
Ein grosses Thema in Sachen Cloud ist die Sicherheit. Die zukünftige Cloud soll dabei nicht zu kurz kommen, weshalb der Webserver mit HTTPS\footnote{\url{http://de.wikipedia.org/wiki/Https}} ausgestattet wird.
\\
Die Folgenden Zeilen müssen dazu kopiert und im Terminal abgesetzt werden:

\begin{itemize}
  \item Zuerst muss ein Zertifikat mit Schlüssel generiert und in ein bestimmtes Verzeichnis abgelegt werden, damit der Webserver es auch finden kann.
    \begin{lstlisting}
sudo mkdir -p /etc/apache2/ssl
sudo openssl req -new -x509 -days 365 -nodes -out /etc/apache2/ssl/apache.pem -keyout /etc/apache2/ssl/apache.pem
sudo ln -sf /etc/apache2/ssl/apache.pem /etc/apache2/ssl/`/usr/bin/openssl x509 -noout -hash < /etc/apache2/ssl/apache.pem`
sudo chmod 600 /etc/apache2/ssl/apache.pem
    \end{lstlisting}
    % FIXME: test if apache listen on port 443 per default
  \item Nach der Erstellung des Zertifikates, muss das Modul geladen werden, welches der Cloud SSL (HTTPS) zur Verfügung stellt:
    \begin{lstlisting}
sudo a2enmod ssl
sudo echo "<virtualhost *:443>
  SSLEngine On
  SSLCertificateFile /etc/apache2/ssl/apache.pem
  DocumentRoot /var/www
</virtualhost>" >> /etc/apache2/sites-available/ssl
sudo a2ensite ssl
    \end{lstlisting}
\end{itemize}

\subsection{PHP konfigurieren}
PHP hat standardmässig ein Sicherheitslimit gesetzt, welches die Uploadgrösse einer Datei einschränkt. Diese soll nun aber erhöht werden, damit der Anwender der Cloud in der Lage ist, Dateien hochzuladen, die grösser als 2 Megabyte sind.

\begin{lstlisting}
sed -i 's/upload_max_filesize = 2M/upload_max_filesize = 2G/' /etc/php5/apache2/php.ini
sed -i 's/post_max_size = 8M/post_max_size = 2G/' /etc/php5/apache2/php.ini
\end{lstlisting}

\subsection{ownCloud installieren}
Zuerst muss die neuste Version von ownCloud heruntergeladen werden. Um herauszufinden, welches die neuste Version ist, besucht man am besten die offizielle Homepage\footnote{\url{https://owncloud.org/}} oder Wikipedia. Zur Zeit ist Version 6.0.1 die neuste. Sie kann wie folgt direkt aus dem Terminal heruntergeladen werden:

% FIXME: check if ~ is preferred directory
\begin{lstlisting}
cd
wget http://download.owncloud.org/community/owncloud-6.0.1.tar.bz2 -O owncloud.tar.bz2
\end{lstlisting}

Nun muss das heruntergeladene Archiv noch entpackt werden und die Dateien in den Ordner ``/var/www'' verschoben werden. Dieser Ordner ist standardmässig der vom Webserver benutzte.

\begin{lstlisting}
tar xvf owncloud.tar.bz2
rm -rf owncloud owncloud.tar.bz2
rm -f /var/www/index.html
sudo mv owncloud/* /var/www
sudo mv owncloud/.htaccess /var/www
\end{lstlisting}

Damit der zuvor erstellte Benutzer ``www-data'' auch auf diese Dateien zugreifen kann, müssen ihm noch die entsprechenden Rechte gegeben werden:

\begin{lstlisting}
sudo chown -R www-data:www-data /var/www
\end{lstlisting}

\subsection{Speicher erweitern}
Da die verwendete SD-Karte eine eher kleine Kapazität hat, lohnt es sich, ownCloud mehr Speicher zur Verfügung zu stellen. Man kann dazu einfach eine externe Festplatte per USB an den Raspberry anschliessen
und diese anschliessend konfigurieren.

\begin{lstlisting}
sudo mkdir -p /mnt/ownclouddata
\end{lstlisting}
Damit die Festplatte bei jedem Neustart wieder automatisch zur Verfügung steht, muss das System dementsprechend konfiguriert werden. Als erstes muss man die Formatierung der angeschlossenen Festplatte ausfindig machen:

\begin{lstlisting}
sudo blkid /dev/sdb1
\end{lstlisting}

Den Wert nach TYPE zwischen den Anführungszeichen muss man sich merken. Es handelt sich um den Namen des verwendeten Dateisystems auf dem Datenträger.
Bevor der nachfolgenden Befehl ausgeführt werden kann, muss das Wort TYPE durch den vorhin gemerkten Wert - das verwendete Dateisystem - ersetzt werden.

\begin{lstlisting}
echo "/dev/sdb1 /mnt/ownclouddata TYPE defaults 0 0" >> /etc/fstab
\end{lstlisting}

Nun muss noch die Festplatte in das System eingebunden und dem Benutzer ``www-data'' die entsprechenden Rechte gegeben werden:

\begin{lstlisting}
sudo mount -a
sudo chown -R www-data:www-data /mnt/ownclouddata
\end{lstlisting}

\subsection{Webserver neustarten}
Zu guter letzt muss der Webserver neu gestartet werden, damit die Änderungen wirksam werden:

\begin{lstlisting}
sudo service apache2 restart
\end{lstlisting}
